\chapter{Trabalhos Relacionados}
\label{cap:trabalhos-relacionados}

Nesta seção, escreva sobre uma introdução sobre os trabalhos relacionados se eles existirem. Evite começar da seção secundária, ou seja, escreva um texto para introduzir as subseções subsequentes, as quais serão sobre cada trabalho relacionado.

\section{Trabalho Relacionado A}
\label{sec:trabalho-relacionado-a}

Insira o texto sobre o Trabalho A aqui:

Texto texto texto texto texto texto texto texto texto texto texto texto texto texto texto texto texto texto texto.

\section{Trabalho Relacionado B}
\label{sec:trabalho-relacionado-b}

Insira o texto sobre o Trabalho B aqui:

Texto texto texto texto texto texto texto texto texto texto texto texto texto texto texto texto texto texto.

Exemplos de inserção de quadro:


	\begin{quadro}[h!]
		\centering
		\Caption{\label{qua:exemplo-1} Praesent ex velit, pulvinar at massa vel, fermentum dictum mauris. Ut feugiat accumsan augue}
		\UFCqua{}{
			\begin{tabular}{|c|c|l|l|}
				\hline
				Quisque & pharetra & tempus & vulputate \\
				\hline
				E1 & Complete coverage by a single transcript & Both  & Complete\\
				\hline
				E2 & Complete coverage by more than & Both splice sites & Complete\\
				\hline
				E3 & Partial coverage & Both splice sites & Both \\
				\hline
			\end{tabular}
		}{
			\Fonte{elaborado pelo autor.}
		}
	\end{quadro}



	\begin{quadro}[h!]
		\centering
		\Caption{\label{qua:exemplo-2} Duis faucibus, enim quis tincidunt pellentesque}
		\UFCqua{}{
			\begin{tabular}{|c|c|}
				\hline
				Quisque & pharetra \\
				\hline
				E1 & Complete coverage by a single transcript \\
				\hline
				E2 & Complete coverage by more than \\
				\hline
				E3 & Partial coverage \\
				\hline
				E4 & Partial coverage \\
				\hline
				E5 & Partial coverage \\
				\hline
				E6 & Partial coverage \\
				\hline
				E7 & Partial coverage \\
				\hline
			\end{tabular}
		}{
			\Fonte{elaborado pelo autor.}
		}
	\end{quadro}


Texto  texto texto texto texto.

%\Gls{ambiguidade}
%\Gls{braile}
%\Gls{coerencia}
%\Gls{dialetos}
%\Gls{elipse}
%\Gls{locucao-adjetiva}
%\Gls{modificadores}
%\Gls{paronimos}
%\Gls{sintese}
%\Gls{borboleta}