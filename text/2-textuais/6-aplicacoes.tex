\chapter{Aplicação}
\label{chap:aplicacoes}

\section{Definição de Problema}

O problema abordado é definido de maneira propositalmente vaga devido à
limitações impostas por sigilo industrial.

No seguinte capítulo é realizada uma análise de correntes de motores elétricos
a fim de detectar anomalias em sua operação. Na situação estudada múltiplos
motores de indução são responsáveis pelo funcionamento de uma esteira
industrial que transporta um produto em uma das etapas de sua fabricação. Os
motores estão acoplados à mesma esteira e a corrente é medida de forma
monofásica na barra de uma de suas gaiolas (TODO: confirmar onde é medida a
corrente e se descrição de medição está ok). Detalhes sobre o processo de
fabricação e produto em si não podem ser compartilhadas e não são
particularmente interessantes para a abordagem deste trabalho, que se limita
aos sinais de corrente. Basta esclarecer que em operação normal a esteira opera
em velocidade constante, enquanto sob operação anômala um acúmulo indesejado do
material transportado gera uma diminuição irregular de sua velocidade.

A partir dessa breve descrição do problema podemos delimitar o escopo do
material utilizado na seguinte análise como exclusivamente os sinais de
corrente dos motores. Nota-se que considerável progresso pode ser feito a
partir de análises dos sistemas de controle de motores de indução, como em
~\cite{}. Quanto à fundamentação teórica, serão utilizados conceitos discutidos
ao longo dos últimos capítulos.

A limitação da análise às correntes leva à necessidade da definição de um
evento anômalo a nível deste sinal para que seja possível gerar algum tipo de
procedimento para sua identificação. Essa definição é realizada \emph{a priori}
por meio de uma rotulação do momento em que a anomalia foi percebida pelos
operadores do processo. Dessa maneira o problema pode ser considerado como
supervisionado.

O desenvolvimento a seguir analisa dois grupos de intervalos de correntes
correspondentes aos modos de operação normal e anômalo. Os procedimentos
terão como objetivo expor tendências indicativas dos eventos anômalos.

\subsection{Descrição dos Dados}

- Visualizações bonitinhas
- informações sobre taxa de amostragem
- estatísticas descritivas genéricas?
- comentários sobre tendências e sucatas óbvias
- comentários sobre dificuldade de adquirir os dados

\subsection{Problemas de operacionalização}

A resolução do problema abordado será desenvolvida levando em conta fatores que
dificultariam a implementação de seus procedimentos em tempo real. Supondo que
seja desenvolvida uma aplicação capaz de distinguir entre as classes descritas
no início deste capítulo, sua operacionalização requer que seus procedimentos
sejam capazes de funcionar em tempo real e de forma contínua. Além disso sua
infraestrutura deve ser interpretável, documentada e reprodutível. A partir
destas considerações enumera-se três grandes problemas de operacionalização
que serão abordados neste trabalho:

\begin{enumerate}
    \item \textbf{Funcionamento em tempo real e contínuo}: Quaisquer modelos
    envolvidos devem não somente realizar inferência em tempo hábil mas também
    se adaptar a mudanças nas características dos sinais como regime de
    operação e conteúdo espectral.
    \item \textbf{Reprodutibilidade}: A infraestrutura de processamento de
    sinais, fluxo de dados, modelagem e inferência deve ser absolutamente
    reprodutível para que a solução seja implementável e passível a manuntenção
    ao longo do tempo
    \item \textbf{Documentação}: O código utilizado nos procedimentos
    envolvidos na solução devem ser apropriadamente documentados pelas mesmas
    motivações do anterior.
\end{enumerate}

TODO: como mencionar soluções para os problemas acima?

\section{Análise de Propriedades das Séries Temporais}

Aplicar testes de sazonalidade, tendência, estacionariedade.

Levantar hipóteses sobre possível resolução:
    - decomposição
    - representação tempo frequêcia por meio de DWT?

\section{Decomposição Linear}

\subsection{Modelagem de Tendência}

\subsection{Modelagem de Sazonalidade}

\subsubsection{Padrão Sazonal Médio}

\subsubsection{Variáveis de Fourier}

\subsection{Análise de Resíduos}

\subsubsection{Modelagem de Resíduos}

manter isso? fazer alguma piração de geração de sinal original a partir de
estímulo de ruído branco?? isso seria massa

\section{Representação em Tempo Frequência}

\subsection{Espectrograma}

\subsection{Distribuição de Wigner Ville}

\subsection{Transformada Contínua de Ondaletas}

\subsection{Transformada Discreta de Ondaletas}

\section{\emph{Pipeline} de Processamento de Modelagem}

mostrar pipeline dvc%, talvez até montar um tikz legal e tal

\section{Resultados}

\section{Análise de Estabilidade Residual por meio Diagrama de Polos e Zeros}
