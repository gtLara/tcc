\chapter{Aplicação}
\label{chap:aplicacoes}

\section{Definição de Problema}

- Não pode falar muito devido ao sigilo
- Deixar claro que a intenção é a extração de características

\subsection{Problemas de operacionalização}

- Tem que funcionar em tempo real, qualquer modelo envolvido tem que evoluir
no tempo. Mencionar filtragrem bayesiana como solução. Mencionar que modelos
complexos como os SOTA da literatura dificilmente vão funcionar. linearidade
seria melhor
- Processamento do sinal deve ser facilmente reprodutível. Mencionar DVC como
solução.

\section{Descrição dos Dados}

Mencionar alta frequência, multiplicidade de séries e sla

\section{Análise de Propriedades das Séries Temporais}

Aplicar testes de sazonalidade, tendência, estacionariedade.

Levantar hipóteses sobre possível resolução:
    - decomposição
    - representação tempo frequêcia por meio de DWT?

\section{Decomposição Linear}

\subsection{Modelagem de Tendência}

\subsection{Modelagem de Sazonalidade}

\subsubsection{Padrão Sazonal Médio}

\subsubsection{Variáveis de Fourier}

\subsubsection{Análise de Resíduos}

\subsubsection{Modelagem de Resíduos}

\section{Representação em Tempo Frequência}

\subsection{Espectrograma}

\subsection{Distribuição de Wigner Ville}

\subsection{Transformada Contínua de Ondaletas}

\subsection{Transformada Discreta de Ondaletas}

\section{\emph{Pipeline} de Processamento de Modelagem}

mostrar pipeline dvc, talvez até montar um tikz legal e tal

\section{Resultados}

\section{Análise de Estabilidade Residual por meio Diagrama de Polos e Zeros}
