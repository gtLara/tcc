\chapter{Teoria Espectral Univariada}\label{chap:spectral_analysis}

\section*{Introdução}

O seguinte capítulo discute a representação espectral de séries temporais
univariadas. São exploradas transformações lineares e não lineares.

O desenvolvimento de uma representação espectral para uma realização
teoricamente infinita de um processo estocástico se inicia no reconhecimento da
impossibilidade de uma transformada de Fourier desse tipo de sinal, seguido de
uma apresentação do teorema de Wiener Khinchin. O resultado da representação
desenvolvida, conhecido como densidade de potência espectral, é interpretado.
Em seguida, restringindo a classe de sinais estacionários para realizações de
processos ARMA, conseguimos derivar expressões fechadas para a densidade
de potência espectral, resultado importante em teoria de
estimação~\cite{estimation_theory}.

Em seguida inicia-se uma discussão sobre representações não lineares por meio
de uma generalização natural da função de autocorrelação e subsequentemente a
classe de distribuições de Cohen. O problema geral dessa classe de
representações e a principal proposta para sua resolução nos leva às
transformadas conhecidas como \emph{Smoothed Pseudo Wigner Ville
Distributions} (SPWVD).

\section{Análise Estacionária}

Sinais estacionários no sentido amplo são, por definição
(seção~\ref{sec:stationarity}), sinais de potência. Como a transformada de
Fourier é bem definida apenas para sinais de energia finita sinais estocásticos
estacionários não possuem uma transformada de Fourier no sentido tradicional.
Para desenvolver uma representação espectral desse tipo de sinal é necessário
definir o conceito de densidade de potência espectral e concluir que essa
função é proporcional ao quadrado da magnitude de uma transformada de Fourier
hipotética.

\subsubsection{Densidade de Potência Espectral}

Constatamos inicialmente o teorema de Parseval, em que $F\{\}$ representa a
transformada de Fourier

$$ E = \int^{\infty}_{-\infty} |x(t)|^2 dt = \frac{1}{2\pi} \int^{\infty}_{-\infty} |F\{x(t)\}(\omega)|^2 d\omega $$

Estendendo essa definição para potência de sinal temos

$$ P = \lim_{T \to \infty} \frac{1}{2T}\frac{1}{2\pi} \int^{T}_{-T}|F\{x(t)\}(\omega)|^2 d\omega$$

Note que apesar de $F\{x(t)\}$ não ser bem definida a relação acima ainda é
válida se $|F\{x(t)\}|^2$ for descrita de uma forma diferente, o que será
feito em breve.

A potência de um sinal pode ser reescrita representando a transformada de
Fourier de $x(t)$ por $X(\omega)$ como

$$ P = \lim_{T \to \infty} \frac{1}{2T}\frac{1}{2\pi} \int^{T}_{-T}|X(\omega)|^2 d\omega $$

Onde $\lim_{T \to \infty}\frac{1}{2\pi} \frac{1}{2T} |X(\omega)|² $
é reconhecida como uma função de densidade. A função densidade de potência
espectral é finalmente definida como

$$ S_{x}(\omega) = \lim_{T \to \infty}\frac{1}{2\pi} \frac{1}{2T} |X(\omega)|² $$

O nome dessa função é bem informativo para sua interpretação: $S_{x}(\omega)$
representa a contribuição das componentes de frequência de $x(t)$ localizadas
em $\omega + d\omega$ para a potência do sinal como um todo. A definição de
$|X(\omega)|^2$ necessária para que essa função faça sentido é fornecida pelo
teorema de Wiener-Khinchin.

\subsection{Teorema de Wiener-Khinchin}

O teorema de Wiener Khinchin pode ser desenvolvido da seguinte maneira

$$ |X(\omega)^2| = X(\omega)X^*(\omega) = F(F^{-1}(X(\omega))*(F^{-1}(X^*(\omega))) = F(x(t) * x^*(-t)) = F(x(t) * x(-t))$$

Examinando a parte mais à direita dessa igualdade observamos que a função que
está sendo transformada corresponde à convolução de $x(t)$ com uma versão
espelhada de si mesmo. Isso é precisamente a definição de autocorrelação.
Assumindo ergodicidade podemos agora expressar a magnitude ao quadrado da
transformada de Fourier de $x(t)$ como a transformada de Fourier de sua
função de autocorrelação.

$$|X(\omega)_T|^2 = \frac{1}{2\pi}\int_{-T}^{T} \rho(t)e^{-j \omega t}dt$$

Esse resultado é conhecido como o teorema de Wiener-Khinchin e permite uma
representação espectral bem definida para sinais estocásticos estacionários.

Note que a transformada de Fourier da autocorrelação de um sinal real é
em si puramente real, propriedade consistente com nossa noção de magnitude ao
quadrado.

\subsection{Espectro de um Processo ARMA}

Tomando a magnitude ao quadrado da transformada Z da forma de recorrência
geral de um processo ARMA (~\ref{ssec:arma_l}) obtemos a seguinte função de
transferência

$$ H(z) = \frac{1 + \sum_{i}^{q} b_k z^{-k}}{1 + \sum_{i}^{q} a_k z^{-k}} $$

que é excitada por ruído branco de forma a gerar uma realização de um processo
ARMA. Podemos agora expressar a densidade de potência espectral de um processo
ARMA como

$$ S_{ARMA}(\omega) = |H(z)|^2 S_{\varepsilon} $$

\begin{equation}\label{eq:arma_spectrum}
     S_{ARMA}(\omega) = \frac{\sigma^2 |1 + \sum_{k=1}^{q} b_k e^{-j\omega k}|^2}{2\pi|1 + \sum_{k=1}^{p} a_k e^{-j\omega k}|^2}
\end{equation}

Essa definição é usada como uma forma de estimação paramétrica de espectro:
os parâmetros são inferidos no domínio do tempo e usados pela relação acima
para sugerir um espectro.

Visualizaremos agora o espectro de alguns processos ARMA.

\subsubsection{MA(1)}

\begin{figure}[H]
    \centering
    \includegraphics[scale=0.7]{figures/ma_1_spectrum.png}
    \caption{Espectro de um processo MA(1) com
    $\protect \beta_1 = -0.5$}
    \label{fig:ma_1_spectrum}
\end{figure}

\subsubsection{AR(1)}

\begin{figure}[H]
    \centering
    \includegraphics[scale=0.7]{figures/ar_1_spectrum_1.png}
    \caption{Espectro de um processo AR(1) com
    $\protect \alpha_1 = 0.8$}
    \label{fig:ar_1_spectrum_1}
\end{figure}

\begin{figure}[H]
    \centering
    \includegraphics[scale=0.7]{figures/ar_1_spectrum_2.png}
    \caption{Espectro de um processo AR(1) com
    $\protect \alpha_1 = -0.8$}
    \label{fig:ar_1_spectrum_2}
\end{figure}

\subsubsection{AR(2)}

\begin{figure}[H]
    \centering
    \includegraphics[scale=0.7]{figures/ar_2_spectrum.png}
    \caption{Espectro de um processo AR(2) com
    $\protect \alpha_1 = 0.5$ and $\protect \alpha_2 = -0.25$}
    \label{fig:ar_2_spectrum}
\end{figure}

\subsubsection{ARMA(4, 3)}

\begin{figure}[H]
    \centering
    \includegraphics[scale=0.7]{figures/arma_4_3_spectrum.png}
    \caption{Espectro de um processo ARMA(4, 3)}
    \label{fig:ar_4_3_spectrum}
\end{figure}


\subsection{Efeitos de raízes unitárias em espectros ARMA}

A introdução de raízes unitárias no polinômio de média móvel ou autoregressivo
de um processo ARMA tem claros efeitos em seu conteúdo espectral, como
mencionado na subseção~\ref{ssec:ma_roots}. O impacto dessas operações no
espectro de um processo pode ser compreendido pelas alterações resultantes ao
gráfico de polos e zeros, pela modificação da série temporal ou por uma análise
da expressão espectral resultante.

Uma análise do gráfico de polos e zeros de um sistema ARMA torna evidente o
efeito de integração na função de transferência do sistema: um polo é
introduzido em $e^{j\omega}=0$ de forma a fornecer energia às componentes de
baixa frequência do sinal. Reciprocamente podemos compreender diferenciação,
que é a introdução de um zero em $e^{j\omega}=0$, como a supressão de
componentes de baixa frequência, resultado em uma operação análoga à filtragem
passa-altas.

No domínio do tempo o efeito de diferenciação pode ser observado comparando as
figuras~\ref{fig:ARMA2-1} e~\ref{fig:ARMA2-1-diff}. A série diferenciada
claramente tem mais energia distribuída em torno de componentes de alta
frequência. Isso é de fato verdadeiro para qualquer sinal: diferenciação
no domínio do tempo age de forma a enviesar o conteúdo espectral para altas
frequências. O recíproco também é verdadeiro: integral um sinal tende seu
conteúdo espectral para baixas frequências por meio da introdução de tendências
estocásticas.

TODO: complete this !!!

Finally the most rigorous understanding is achieved by inspection of equation
~\ref{eq:arma_spectrum}. From this equation it is evident that an additional
root on the numerator increases the high frequency content and an additional
root on the denominator increases the low frequency content.

We now present the differenced version of the MA(1) spectrum from figure
~\ref{fig:ma_1_spectrum}. Note that the signal is essentially high pass
filtered, as mentioned.

\subsubsection{Differenced MA(1)}

\begin{figure}[H]
    \centering
    \includegraphics[scale=0.7]{figures/diff_ma_1_spectrum.png}
    \caption{Spectrum of an differenced MA(1) process with
    $\protect \beta_1 = -0.5$}
    \label{fig:diff_ma_1_spectrum}
\end{figure}


\section{Linear Non-Stationary Analysis}

From the previous section we can conclude that if a time series is stationary
its spectral representation via power spectral density is uniquely determined
by the Fourier transform of its autocorrelation function. The natural extension
for a spectral representation of non-stationary processes is a time-varying
power spectral density since its autocorrelation function is also time-varying.

The idea of a time-varying spectral representation gives rise to the so called
time-frequency analysis methods. We initially explore methods that maintain
linearity.

\section{Non-Linear Representations}

There is, indeed, a way to maximize the time-frequency resolution trade-off
inherent to time-frequency representations (TFR). This is done by the
introduction of non linearity through the Fourier transform of the
instantaneous autocorrelation function. As will be seen presence of non
linearity results in cross terms that limit the quality of the representation.
Attempts to dampen these cross terms lead to the more general Cohen's class of
distributions.

\subsection{Instantaneous autocorrelation function}

The instantaneous autocorrelation function is actually just the autocorrelation
function of a non stationary signal written in a specific format.
Interestingly the term autocorrelation function has become strongly understood
as a function of sample lag $\tau$, which is the case for stationary signals,
instead of a function of $t_1$ and $t_2$. We initially rewrite the general
autocorrelation function $R_{xx}$ of a signal $x(t)$

$$ R_{xx}(t_1, t_2) = E[x(t_1)x(t_2)] $$

We can also write $R_xx$ as a function of a single moment $t$ and a lag $\tau$

$$ R_{x}(t, \tau) = E[x(t)x(t-\tau)] $$

Which is slightly more natural for computations. Note that if $x(t)$ is
stationary the dependence on time is removed because $R_{xx}$ has the same
value for all $t$.

A small adjustment in notation leads to

$$ \mathcal{R}_{x}(t, \tau) = E\left[x\left(t - \frac{\tau}{2}\right)x\left(t + \frac{\tau}{2}\right)\right] $$

With a new symbol to indicate that we have finally arrived at the instantaneous
autocorrelation function $\mathcal{R}_{xx}$.

A non stationary TFR is now natural. Since the
Wiener-Khinchin theorem states that the spectrum of a stationary signal is the
Fourier transform of its autocorrelation function we can in an analogous manner
assume that a spectral representation of a non stationary process will be given
by the Fourier transform along the $\tau$ axis of the instantaneous
autocorrelation function. This leads to the Wigner-Ville distribution.

\subsection{Wigner-Ville Distribution}

We define the Wigner-Ville distribution (WVD) as

$$ \mathcal{W}_{x}(t, f) =  \int_{-\infty}^{\infty} \mathcal{R}_{x}(t, \tau) e^{-j\omega \tau}d\tau$$

$$ \mathcal{W}_{x}(t, f) =  \int_{-\infty}^{\infty} x\left(t - \frac{\tau}{2}\right)x\left(t + \frac{\tau}{2}\right) e^{-j\omega \tau}d\tau$$

This natural representation can be thought of as an instantaneous power density
spectrum. It is known that the WVD optimizes the time-frequency resolution
trade-off\cite{tfr_comparison} which is exactly our goal in the development of
more elaborated TFRs. We will now see that this is not without its problems.

\subsubsection{Cross-terms}

By construction the Wigner-Ville distribution is a quadratic representation. By
the quadratic superposition principle~\cite{quadratic_freq_representation} we
know that if $x(t) = \mu x_1(t) + \lambda x_2(t)$ the WVD representation of
$x(t)$ will be given by

$$ \mathcal{W}_{x} = \mu^2\mathcal{W}_{x_1} + \lambda^2\mathcal{W}_{x_2} + 2(\lambda\mu)^2(\mathcal{W}_{x_1 , x_2})$$

Where $\mathcal{W}_{z, y}$ represents the cross-WVD from $z$ to $y$. Since any
real signal of relevant complexity is a linear combination of the $cos(t)$
$sin(t)$ basis we can expect a considerable introduction of unwanted
information from the cross-WVD components, referred to as cross-terms. This is
the infamous cross-terms problem attributed to the WVD and is one of the
reasons that despite its precision in time and frequency resolution it is not
an ideal choice for the TFR for most signals.

The cross-terms are known to exhibit high frequency patterns~\cite{martin_lol},
leading to the idea that the WVD could be filtered in order to be more
representative of its auto-terms. The different ways in which it is possible
and useful to filter the WVD generates what is known as the Cohen's class of
distributions.

\subsection{Smoothed Pseudo Wigner Ville Distributions}

Most members of Cohen's class of distributions are essentially filtered
versions of the WVD~\cite{}. A particularly useful case is known as the
Smoothed Pseudo Wigner Ville Distribution (SPWVD)

$$ SPWVD(t, f) = \int_{-\infty}^{\infty}\int_{-\infty}^{\infty} h_t (t - \tau) h_f(f - \phi)d\tau d\phi$$

in which $h_t$ denotes the filter applied along time and $h_f$ along frequency.

If the filters are well designed, which is a data-driven process in some
cases~\cite{}, cross-term suppression is sufficiently successful such that
the use of the SPWVD over simpler non-stationary TFR is justified.
