\chapter{Testes Estatísticos}

\section{Considerações sobre testes em séries temporais}

\section{Teste de \emph{Dickey-Fuller}}

Os testes de \emph{Dickey-Fuller} testam a hipótese nula de presença de raízes
unitárias no processo gerador de uma série temporal com a hipótese alternativa
de estacionariedade.

O teste mais simples de \emph{Dickey-Fuller} assume que o processo gerador da
série temporal em questão é dado pela equação~\ref{eq:ad}, na qual
$\varepsilon$ é ruído branco.

\begin{equation}\label{eq:ad}
    y_t = \phi y_{t-1} + \varepsilon_t
\end{equation}

A hipótese nula do teste é a presença de raízes unitárias em $\mathbf{y}_t$,
isso é, $\phi = 1$, e a hipótese alternativa é $\phi < 1$, correspondente à
estacionariedade.

$$
\begin{cases}
    H_0: \phi = 1, \text{não estacionariedade (raízes unitárias, possível diferença-estacionariedade)} \\
    H_1: \phi < 1, \text{estacionariedade}
\end{cases}
$$

Em seguida $Ly_t$ é subtraído de ambos os lados da equação~\ref{eq:ad},
resultando no desenvolvimento a seguir:

$$ y_t - y_{t-1} = \phi y_{t-1} - y_{t-1} + \varepsilon $$
$$ \nabla y_t = (\phi - 1) y_{t-1}  + \varepsilon $$
$$ \nabla y_t = \delta y_t  + \varepsilon $$

As hipóteses do teste são agora reformuladas para as seguintes

$$
\begin{cases}
    H_0: \delta = 0, \text{não estacionariedade (raízes unitárias, diferença-estacionariedade)} \\
    H_1: \delta < 0, \text{estacionariedade}
\end{cases}
$$

TODO: definir estatística de teste

A estatística de teste é computada e comparada com um valor crítico proveniente
da distribuição de \emph{Dickey-Fuller} (geralmente sob $p=0.5$) para rejeição
ou não da hipótese nula.

O modelo da série temporal, dado explicitamente pela equação~\ref{eq:ad}, pode
ser alterado para testar raízes unitárias com constante e com constante e
tendência determinística no tempo por meio das equações~\ref{eq:ad_constant} e
~\ref{eq:ad_constant_trend}, respectivamente. Note que ambas as equações são
apresentadas em função de $\delta$. É mais comum testar por apenas raízes
unitárias, uma vez que uma análise subjetiva visual, por correlalograma ou
decomposição juntamente de remoção de tendência ou sazonalidade é tipicamente
realizada antes de um teste estatístico de estacionariedade.

\begin{equation}\label{eq:ad_constant}
    \nabla y_t = \delta y_{t-1} + u_t + a_0
\end{equation}

\begin{equation}\label{eq:ad_constant_trend}
    \nabla y_t = \delta y_{t-1} + u_t + a_0 + a_1 t
\end{equation}

O teste aumentado de \emph{Dickey-Fuller}(ADF) modela o processo de forma mais
geral, incluindo na equação~\ref{eq:ad} termos representativos de processos
estacionários arbitrários. O teste aumentado é projetado para remover
autocorrelação do processo de validação de hipótese. De forma identicamente
análoga ao teste de DF temos expansões do ADF para incluir constantes e
tendências como nas equações~\ref{eq:ad_constant} e
~\ref{eq:ad_constant_trend} apesar de que, como no teste de DF, essas
variações são pouco usadas. A estatística de teste do ADF é negativa, isso é,
quanto menor seu valor maior a rejeição da hipótese nula de não
estacionariedade (maior certeza de estacionariedade).

Como em qualquer teste de hipótese um valor $p$ maior que $0.05$ indica falha
em rejeitar a hipótese nula, nesse caso correspondendo à impossibilidade de
constatar estacionariedade. Um valor $p$ menor ou igual a $0.05$ indica
rejeição da hipótese nula, correspondendo à conclusão que a série sob análise
é estacionária.

Na prática o teste mais usado é o ADF que é uma simples extensão mais robusta
do teste de \emph{Dickey-Fuller}. Implementações eficiente e populares existem
para \verb+R+ e \verb+Python+.

\subsection{Teste \emph{Kwiatkowski-Phillips-Schmidt-Shin} (KPSS)}

TODO: check this

O teste KPSS desempenha uma função semelhante ao ADF com a relevante
diferença de inerentemente modelar uma tendência linear no tempo por meio da
equação~\ref{eq:KPSS}. Seu desenvolvimento matemático é análogo porém mais
trabalhoso que o caso do ADF e será portanto omitido.

\begin{equation}\label{eq:KPSS}
    y_t = \phi y_{t-1} + \varepsilon_t + \beta t
\end{equation}

Com $\varepsilon_t$ representando ruído branco. O teste em seguida define as
seguintes hipóteses:

$$
\begin{cases}
    H_0: \text{a série apresenta tendência-estacionariedade} \\
    H_1: \text{a série apresenta raízes unitárias}
\end{cases}
$$

Como em qualquer teste de hipótese um valor $p$ maior que $0.05$ indica falha
em rejeitar a hipótese nula, nesse caso correspondendo à impossibilidade de
constatar que a série não apresenta tendência estacionariedade, boa evidência
de que a série é tendência-estacionária. Um valor $p$ menor ou igual a $0.05$
indica rejeição da hipótese nula, correspondendo à conclusão que a série sob
análise possui raízes unitárias e é portanto não estacionária.

Observe que há uma diferença crítica: a alternativa nula não postula não
estacionariedade, como no caso do ADF, mas sim tendência-estacionariedade
(seção ~\ref{ssec:taxonomy}) decorrente diretamente da inclusão de tendência
linear no modelo da equação~\ref{eq:KPSS}. A diferença principal da alteração
da hipótese nula é que o KPSS é usado para investigar presença de
estacionariedade sob uma tendência determinística (tendência-estacionariedade)
e o ADF (tipicamente) de estacionariedade propriamente dita. O KPSS é bem
implementado em \verb+R+ e \verb+Python+.

\section{Causalidade de Granger}

\section{Box-Pierce}

\section{Ljung-Box-Pierce}

\section{\emph{Convergence Cross Mapping}}

\section{Johansen}

\section{Considerações sobre testes em séries temporais}

\section{Teste de \emph{Dickey-Fuller}}

Os testes de \emph{Dickey-Fuller} testam a hipótese nula de presença de raízes
unitárias no processo gerador de uma série temporal com a hipótese alternativa
de estacionariedade.

O teste mais simples de \emph{Dickey-Fuller} assume que o processo gerador da
série temporal em questão é dado pela equação~\ref{eq:ad}, na qual
$\varepsilon$ é ruído branco.

\begin{equation}\label{eq:ad}
    y_t = \phi y_{t-1} + \varepsilon_t
\end{equation}

A hipótese nula do teste é a presença de raízes unitárias em $\mathbf{y}_t$,
isso é, $\phi = 1$, e a hipótese alternativa é $\phi < 1$, correspondente à
estacionariedade.

$$
\begin{cases}
    H_0: \phi = 1, \text{não estacionariedade (raízes unitárias, possível diferença-estacionariedade)} \\
    H_1: \phi < 1, \text{estacionariedade}
\end{cases}
$$

Em seguida $Ly_t$ é subtraído de ambos os lados da equação~\ref{eq:ad},
resultando no desenvolvimento a seguir:

$$ y_t - y_{t-1} = \phi y_{t-1} - y_{t-1} + \varepsilon $$
$$ \nabla y_t = (\phi - 1) y_{t-1}  + \varepsilon $$
$$ \nabla y_t = \delta y_t  + \varepsilon $$

As hipóteses do teste são agora reformuladas para as seguintes

$$
\begin{cases}
    H_0: \delta = 0, \text{não estacionariedade (raízes unitárias, diferença-estacionariedade)} \\
    H_1: \delta < 0, \text{estacionariedade}
\end{cases}
$$

TODO: definir estatística de teste

A estatística de teste é computada e comparada com um valor crítico proveniente
da distribuição de \emph{Dickey-Fuller} (geralmente sob $p=0.5$) para rejeição
ou não da hipótese nula.

O modelo da série temporal, dado explicitamente pela equação~\ref{eq:ad}, pode
ser alterado para testar raízes unitárias com constante e com constante e
tendência determinística no tempo por meio das equações~\ref{eq:ad_constant} e
~\ref{eq:ad_constant_trend}, respectivamente. Note que ambas as equações são
apresentadas em função de $\delta$. É mais comum testar por apenas raízes
unitárias, uma vez que uma análise subjetiva visual, por correlalograma ou
decomposição juntamente de remoção de tendência ou sazonalidade é tipicamente
realizada antes de um teste estatístico de estacionariedade.

\begin{equation}\label{eq:ad_constant}
    \nabla y_t = \delta y_{t-1} + u_t + a_0
\end{equation}

\begin{equation}\label{eq:ad_constant_trend}
    \nabla y_t = \delta y_{t-1} + u_t + a_0 + a_1 t
\end{equation}

O teste aumentado de \emph{Dickey-Fuller}(ADF) modela o processo de forma mais
geral, incluindo na equação~\ref{eq:ad} termos representativos de processos
estacionários arbitrários. O teste aumentado é projetado para remover
autocorrelação do processo de validação de hipótese. De forma identicamente
análoga ao teste de DF temos expansões do ADF para incluir constantes e
tendências como nas equações~\ref{eq:ad_constant} e
~\ref{eq:ad_constant_trend} apesar de que, como no teste de DF, essas
variações são pouco usadas. A estatística de teste do ADF é negativa, isso é,
quanto menor seu valor maior a rejeição da hipótese nula de não
estacionariedade (maior certeza de estacionariedade).

Como em qualquer teste de hipótese um valor $p$ maior que $0.05$ indica falha
em rejeitar a hipótese nula, nesse caso correspondendo à impossibilidade de
constatar estacionariedade. Um valor $p$ menor ou igual a $0.05$ indica
rejeição da hipótese nula, correspondendo à conclusão que a série sob análise
é estacionária.

Na prática o teste mais usado é o ADF que é uma simples extensão mais robusta
do teste de \emph{Dickey-Fuller}. Implementações eficiente e populares existem
para \verb+R+ e \verb+Python+.

\subsection{Teste \emph{Kwiatkowski-Phillips-Schmidt-Shin} (KPSS)}

TODO: check this

O teste KPSS desempenha uma função semelhante ao ADF com a relevante
diferença de inerentemente modelar uma tendência linear no tempo por meio da
equação~\ref{eq:KPSS}. Seu desenvolvimento matemático é análogo porém mais
trabalhoso que o caso do ADF e será portanto omitido.

\begin{equation}\label{eq:KPSS}
    y_t = \phi y_{t-1} + \varepsilon_t + \beta t
\end{equation}

Com $\varepsilon_t$ representando ruído branco. O teste em seguida define as
seguintes hipóteses:

$$
\begin{cases}
    H_0: \text{a série apresenta tendência-estacionariedade} \\
    H_1: \text{a série apresenta raízes unitárias}
\end{cases}
$$

Como em qualquer teste de hipótese um valor $p$ maior que $0.05$ indica falha
em rejeitar a hipótese nula, nesse caso correspondendo à impossibilidade de
constatar que a série não apresenta tendência estacionariedade, boa evidência
de que a série é tendência-estacionária. Um valor $p$ menor ou igual a $0.05$
indica rejeição da hipótese nula, correspondendo à conclusão que a série sob
análise possui raízes unitárias e é portanto não estacionária.

Observe que há uma diferença crítica: a alternativa nula não postula não
estacionariedade, como no caso do ADF, mas sim tendência-estacionariedade
(seção ~\ref{ssec:taxonomy}) decorrente diretamente da inclusão de tendência
linear no modelo da equação~\ref{eq:KPSS}. A diferença principal da alteração
da hipótese nula é que o KPSS é usado para investigar presença de
estacionariedade sob uma tendência determinística (tendência-estacionariedade)
e o ADF (tipicamente) de estacionariedade propriamente dita. O KPSS é bem
implementado em \verb+R+ e \verb+Python+.

\section{Causalidade de Granger}

\section{Box-Pierce}

\section{Ljung-Box-Pierce}

\section{\emph{Convergence Cross Mapping}}

\section{Johansen}
