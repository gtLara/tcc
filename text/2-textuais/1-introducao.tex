\chapter{Introdução}
\label{cap:introducao}

Séries temporais como uma categoria de dados ganhou tremenda importância no
advento de \emph{big data} por representar essencialmente qualquer tipo de
informação que evolui no tempo. Grande parte da teoria envolvida no campo
conhecido como análise de séries temporais é idêntica ou muito semelhante a
conceitos de sistemas lineares e processamento de sinais, mas a compreensão dos
temas na literatura científica e didática atual frequentemente não explora o
potencial analítico dessa interseção.
TODO: mention stochastic signal processing

O seguinte trabalho tem por objetivo abordar teoria clássica de séries
temporais por um ponto de vista de sinais e sistemas. Em uma extensa revisão de
literatura cada definição é introduzida notando paralelos em nomenclatura,
conceito e papel analítico entre os dois campos. Eventualmente conceitos
dificilmente capturados por quaisquer uma das áreas isoladamente são abordados
de forma integrada e a simplicidade resultante é notável. Em uma reflexão
teórica conclusiva o limite de comunicação dos campos é traçado. Por fim uma
aplicação prática usando o corpo teórico desenvolvido é apresentada, na qual
uma série de análises atípicas e informativas são realizadas por meio da
abordagem desenvolvida.

O capítulo 2 inicia com a apresentação de uma base de definições e
nomenclatura, sempre explorando os paralelos entre os campos abordados. Ao
final do capítulo o conceito crítico de estacionariedade é apresentado e
interpretado com devida atenção.

O capítulo 3 introduz teoria de decomposição, modelagem ARMA, modelagem sazonal
e contém a interpretação mais direta de séries temporais como saídas de
sistemas lineares. Essa interpretação é desenvolvida de forma a delimitar até
onde as duas áreas podem se comunicar de forma a produzir análises úteis.

O capítulo 4 discute a representação espectral de séries temporais univariadas.
São exploradas transformações lineares e não lineares. O desenvolvimento de uma
representação espectral para uma realização teoricamente infinita de um
processo estocástico é apresentada. Em seguida inicia-se uma discussão sobre
representações não lineares por meio de uma generalização natural da função de
autocorrelação e subsequentemente a classe de distribuições de Cohen.

O capítulo 5 brevemente discute testes estatísticos utilizados no capítulo 6.

O capítulo 6 apresenta um problema de extração de características para detecção
de anomalias em uma série temporal multivariada com sazonalidade elaborada. O
problema é abordado de forma prática - preocupações sobre sua resolução no
mundo real são levantadas e ferramentas de operacionalização de modelos são
utilizadas.
