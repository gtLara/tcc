\chapter{Univariate Spectral Theory}

\section*{Introduction}

The following chapter discusses the spectral representation of univariate time
series, stationary and non-stationary, linear and non-linear. Some important
aspects of multivariate spectral analysis are developed in
chapter~\ref{chap:multi}.

The development of a spectral representation for the theoretical infinite
realization of a stationary stochastic process begins at the recognition of the
impossibility of a Fourier for this type of signal followed by the introduction
of the Wiener Khinchin theorem. After an interpretation of the resulting
spectral representation (the power spectral density) some deterministic linear
system theory leads to the idea of a shaping filter, which finally allows us to
deduce a general expression for the power spectral density of ARMA processes.

We then proceed to linear but non-stationary analysis, in which the short time
Fourier transform and its shortcomings are presented as motivation for the
S-transform. Wavelet analysis is then introduced motivated by the demand for a
mathematically cleaner version of the S-transform, leading to the concepts of
continuous and discrete wavelet transforms. On the topic of resolution the
possibility of an optimized time-frequency resolution is suggested.

Non-linear analysis is constructed by the introduction of the instantaneous
auto-correlation function and subsequent Cohen's class of distributions. Some
main distributions from Cohens class are discussed along with their
limitations, concluding our univariate spectral exploration.

\section{Stationary Analysis}

Wide sense stationary signals are, by definition (section
~\ref{sec:stationarity}), power signals. Since the Fourier transform of a
signal is well defined only if is has finite energy stationary signals do not
have a Fourier representation in the traditional sense, the exception being
quasi-periodic signals which can be represented by a Fourier series. In order
to develop a spectral representation of stationary time series we must define
the concept of a power spectral density and conclude that this function is
proportional to the square magnitude of a hypothetical Fourier transform.

\subsubsection{Power Spectral Density}

We begin by stating Parseval's theorem, in which $F\{\}$ represents the Fourier
transform.

$$ E = \int^{\infty}_{-\infty} |x(t)|^2 dt = \frac{1}{2\pi} \int^{\infty}_{-\infty} |F\{x(t)\}(\omega)|^2 d\omega $$

Extending this definition to signal power gives us

$$ P = \lim_{T \to \infty} \frac{1}{2T}\frac{1}{2\pi} \int^{T}_{-T}|F\{x(t)\}(\omega)|^2 d\omega$$

Note that even though $F\{x(t)\}$ is not well defined here the above relations
still hold if $|F\{x(t)\}|^2$ is defined in a different manner, which will be
done shortly.

The signal power can be rewritten denoting the Fourier transform of $x(t)$ by
$X(\omega)$ as

$$ P = \lim_{T \to \infty} \frac{1}{2T}\frac{1}{2\pi} \int^{T}_{-T}|X(\omega)|^2 d\omega $$

Where $\lim_{T \to \infty}\frac{1}{2\pi} \frac{1}{2T} |X(\omega)|² $
is recognized as a density function. The power spectral density function is
finally defined as

$$ S_{x}(\omega) = \lim_{T \to \infty}\frac{1}{2\pi} \frac{1}{2T} |X(\omega)|² $$

This function's name is pretty explanatory of its interpretation:
$S_{x}(\omega)$ represents the contribution of $x(t)$s frequency components in
$\omega + d\omega$ to the overall signal power. As mentioned, for this function
to make any sense we must define $|X(\omega)|^2$, which will be done
presently

\subsection{Wiener-Khinchin Theorem}

The Wiener-Khinchin theorem can be developed as follows.

$$ |X(\omega)^2| = X(\omega)X^*(\omega) = F(F^{-1}(X(\omega))*(F^{-1}(X^*(\omega))) = F(x(t) * x^*(-t)) = F(x(t) * x(-t))$$

Examining the right-most part of this equality we observe that the function
which is being Fourier-transformed corresponds to the convolution of $x(t)$
with a mirrored version of itself. This is precisely the definition of
autocorrelation. Assuming ergodicity we can now express the squared magnitude
of the Fourier transform of $x(t)$ as the Fourier transform of is autocorrelation
function.

$$|X(\omega)_T|^2 = \frac{1}{2\pi}\int_{-T}^{T} \rho(t)e^{-j \omega t}dt$$

This results is known as the Wiener-Khinchin theorem and it allows for a
well-defined power density spectrum for stochastic signals.

Note that since the autocorrelation of a signal is even its Fourier transform
is real-valued, which is consistent with our notion of a squared magnitude.

\subsection{Spectrum of ARMA processes}

By taking the square magnitude Z transform of the general ARMA recurrence
relationship (~\ref{ssec:arma_l}) we obtain the transfer function

$$ H(z) = \frac{1 + \sum_{i}^{q} b_k z^{-k}}{1 + \sum_{i}^{q} a_k z^{-k}} $$

which is excited by white noise to generate a realization of and ARMA
process. We can now express the power spectral density of an ARMA process as
follows

$$ S_{ARMA}(\omega) = |H(z)|^2 S_{\varepsilon} $$
$$ S_{ARMA}(\omega) = \frac{\sigma^2 |1 + \sum_{k=1}^{q} b_k e^{-j\omega k}|^2}{2\pi|1 + \sum_{k=1}^{p} a_k e^{-j\omega k}|^2} $$

This definition can be used as a means of parametric spectral estimation: the
parameters are estimated in the time domain and used by the relationship
above to estimate the spectrum.

We will now visualize the spectra of some ARMA processes alongside their
autocorrelation functions. Note how the notion that the spectra corresponds to
the Fourier transform of the autocorrelation function is intuitive.

\subsubsection{ARMA(0, 1)}

\subsubsection{ARMA(2, 0)}

\subsubsection{ARMA(2, 3)}

\section{Non-Stationary Analysis}

\subsection{Short Time Fourier Transform}

\subsection{S-Transform}

Drop this maybe

\subsection{Wavelet Transform}

\subsubsection{Continuous}

\subsubsection{Discrete}

\section{Non-Linear Representations}

There is, indeed, a way to maximize the time-frequency resolution trade-off
inherent to time-frequency representations\cite{HERE BRO}. This is done by the introduction
of non linearity through the Fourier transform of the instantaneous
autocorrelation function. As will be seen presence of non linearity results in
cross terms that limit the quality of the representation. Attempts to dampen
these cross terms lead to the more general Cohen's class of distributions.

\subsection{Instantaneous autocorrelation function}

\subsection{Wigner-Wille Distribution}

Auto and Crossterms

\subsection{Cohen's class}

\subsubsection{Ambiguity Function}

\subsubsection{Ambiguity Function}

\subsubsection{Ambiguity Function}
