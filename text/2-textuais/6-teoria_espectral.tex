\chapter{Teoria Espectral Univariada}

\section*{Introduction}

The following chapter discusses the spectral representation of univariate time
series, stationary and non-stationary, linear and non-linear. Some important
aspects of multivariate spectral analysis are developed in
chapter~\ref{chap:multi}.

The development of a spectral representation for the theoretical infinite
realization of a stationary stochastic process begins at the recognition of the
impossibility of a Fourier for this type of signal followed by the introduction
of the Wiener Khinchin theorem. After an interpretation of the resulting
spectral representation (the power spectral density) some deterministic linear
system theory leads to the idea of a shaping filter, which finally allows us to
deduce a general expression for the power spectral density of ARMA processes.

We then proceed to linear but non-stationary analysis, in which the short time
Fourier transform and its shortcomings are presented as motivation for the
S-transform. Wavelet analysis is then introduced motivated by the demand for a
mathematically cleaner version of the S-transform, leading to the concepts of
continuous and discrete wavelet transforms. On the topic of resolution the
possibility of an optimized time-frequency resolution is suggested.

Non-linear analysis is constructed by the introduction of the instantaneous
auto-correlation function and subsequent Cohen's class of distributions. Some
main distributions from Cohens class are discussed along with their
limitations, concluding our univariate spectral exploration.
