%http://www.biblioteca.ufc.br/wp-content/uploads/2015/09/tutorial-sharelatex.pdf
\usepackage[utf8]{inputenc}                         % Acentuação direta
\usepackage[T1]{fontenc}                            % Codificação da fonte em 8 bits
\usepackage{graphicx}                               % Inserir figuras
\usepackage{amsfonts, amssymb, amsmath}             % Fonte e símbolos matemáticos
\usepackage{booktabs}                               % Comandos para tabelas
\usepackage{verbatim}                               % Texto é interpretado como escrito no documento
\usepackage{multirow, array}                        % Múltiplas linhas e colunas em tabelas
\usepackage{indentfirst}                            % Endenta o primeiro parágrafo de cada seção.
\usepackage{listings}                               % Utilizar codigo fonte no documento
\usepackage{xcolor}
\usepackage{microtype}                              % Para melhorias de justificação?
\usepackage[portuguese,ruled,lined]{algorithm2e}    % Escrever algoritmos
\usepackage{algorithmic}                            % Criar Algoritmos
\usepackage{float}                                  % Utilizado para criação de floats
\usepackage{amsgen}
%\usepackage{titlesec}                              % Permite alterar os títulos do documento
\usepackage{tocloft}                                % Permite alterar a formatação do Sumário
\usepackage{etoolbox}                               % Usado para alterar a fonte da Section no Sumário
\usepackage{tikz}
\usetikzlibrary{arrows}
\usepackage[nogroupskip,nonumberlist]{glossaries}   % Permite fazer o glossario

\usepackage[font=singlespacing,format=plain,justification=justified,skip=0pt,singlelinecheck = false]{caption}            % Altera o comportamento da tag caption

\usepackage[alf, abnt-emphasize=bf, recuo=0cm, abnt-etal-cite=2, abnt-etal-list=0, abnt-etal-text=it]{abntex2cite}  % Citações padrão ABNT
\usepackage[bottom]{footmisc}                      % Mantém as notas de rodapé sempre na mesma posição
% \usepackage{times}                                 % Usa a fonte Times
%%%%%%%%%%%%%%%%%%%%%%%%%%%%%%%%%%%%%%%%%%%%%%%%%%%%%%%%%%%%%%%%%%

\usepackage{mathptmx}         % Usa a fonte Times New Roman			%\usepackage{lmodern}         % Usa a fonte Latin Modern
%\usepackage{subfig}          % Posicionamento de figuras
%\usepackage{scalefnt}        % Permite redimensionar tamanho da fonte
%\usepackage{color, colortbl} % Comandos de cores
%\usepackage{lscape}          % Permite páginas em modo "paisagem"
%\usepackage{ae, aecompl}     % Fontes de alta qualidade
%\usepackage{picinpar}        % Dispor imagens em parágrafos
%\usepackage{latexsym}        % Símbolos matemáticos
%\usepackage{upgreek}         % Fonte letras gregas
\usepackage{appendix}         % Gerar o apendice no final do documento
\usepackage{paracol}          % Criar paragrafos sem identacao
\usepackage{text/lib/ufctex}	      % Biblioteca com as normas da UFC para trabalhos academicos
\usepackage{pdfpages}         % Incluir pdf no documento
\usepackage{amsmath}          % Usar equacoes matematicas

\makeglossaries % Organiza e gera a lista de abreviaturas, simbolos e glossario
\makeindex      % Gera o Indice do documento
