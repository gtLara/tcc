\documentclass[
    a4paper,          % Tamanho da folha A4
    12pt,             % Tamanho da fonte 12pt
    chapter=TITLE,    % Todos os capitulos devem ter caixa alta
    section=Title,    % Todas as secoes devem ter caixa alta somente na primeira letra
    subsection=Title, % Todas as subsecoes devem ter caixa alta somente na primeira letra
    oneside,          % Usada para impressao em apenas uma face do papel
    english,          % Hifenizacoes em ingles
    spanish,          % Hifenizacoes em espanhol
    brazil           % Ultimo idioma eh o idioma padrao do documento
   % fleqn             % Comente esta linha se quiser centralizar as equacoes. Comente também a linha 65 abaixo
]{abntex2}

\input{text/lib/preambulo}

\trabalhoacademico{tccgraduacao}

%%%%%%%%%%%%%%%%%%%%%%%%%%%%%%%%%%%%%%%%%%%%%%%%%%%%%

% Define se o trabalho e uma qualificacao
% Coloque 'nao' para versao final do trabalho

\ehqualificacao{nao}

% Remove as bordas vermelhas e verdes do PDF gerado
% Coloque 'sim' pare remover

\removerbordasdohyperlink{sim}

% Adiciona a cor Azul a todos os hyperlinks

\cordohyperlink{nao}

%%%%%%%%%%%%%%%%%%%%%%%%%%%%%%%%%%%%%%%%%%%%%%%%%%%%%
%%         Informacao sobre a instituicao          %%
%%%%%%%%%%%%%%%%%%%%%%%%%%%%%%%%%%%%%%%%%%%%%%%%%%%%%

\ies{Universidade Federal de Minas Gerais}
\iessigla{UFMG}
\centro{}
\departamento{Departamento de Engenharia Eletrônica}

\graduacaoem{Engenharia Elétrica}
\habilitacao{Bacharelado}

% AVISO: Caso necessario alterar o texto de apresenta-
% cao da Especializacao, ir a pasta "lib", arquivo
% "ufctex.sty" na linha 512.

% AVISO: Caso necessario alterar o texto de apresenta-
% cao da Especializacao, ir a pasta "lib", arquivo
% "ufctex.sty" na linha 517.


%%%%%%%%%%%%%%%%%%%%%%%%%%%%%%%%%%%%%%%%%%%%%%%%%%%%%
%%      Informacoes relacionadas ao trabalho       %%
%%%%%%%%%%%%%%%%%%%%%%%%%%%%%%%%%%%%%%%%%%%%%%%%%%%%%

\autor{Gabriel Teixeira Lara Chaves}
\titulo{Uma abordagem de sistemas lineares a teoria espectral
de séries temporais estacionárias e não estacionárias}
\data{2023}
\local{Belo Horizonte}

% Exemplo: \dataaprovacao{6th September 2019}
\dataaprovacao{\today}

%%%%%%%%%%%%%%%%%%%%%%%%%%%%%%%%%%%%%%%%%%%%%%%%%%%%%
%%           Informação sobre o Orientador         %%
%%%%%%%%%%%%%%%%%%%%%%%%%%%%%%%%%%%%%%%%%%%%%%%%%%%%%

\orientador{Prof. Dr. Frederico Gualberto Coelho}
\orientadories{Universidade Federal de Minas Gerais}
\orientadorfeminino{sim} % Coloque 'sim' se for do sexo feminino

%%%%%%%%%%%%%%%%%%%%%%%%%%%%%%%%%%%%%%%%%%%%%%%%%%%%%
%%          Informação sobre o Coorientador        %%
%%%%%%%%%%%%%%%%%%%%%%%%%%%%%%%%%%%%%%%%%%%%%%%%%%%%%

% Deixe o nome do coorientador em branco para remover do documento

\coorientador{}
\coorientadories{Universidade Federal de Minas Gerais}
% \coorientadorcentro{Centro do Coorientador (SIGLA)}
\coorientadorfeminino{nao} % Coloque 'sim' se for do sexo feminino

%%%%%%%%%%%%%%%%%%%%%%%%%%%%%%%%%%%%%%%%%%%%%%%%%%%%%
%%              Informação sobre a banca           %%
%%%%%%%%%%%%%%%%%%%%%%%%%%%%%%%%%%%%%%%%%%%%%%%%%%%%%

\membrodabancadois{Prof. Dr. Adriano Vilela Barbosa}
\coorientadories{Universidade Federal de Minas Gerais}

\begin{document}

    % \selectlanguage{portuguese} % Colocando em inglês
    \bibliographystyle{text/lib/abntex2-alf} % Ajustes das referências em inglês

	% Elementos pré-textuais
    \imprimircapatcc{Undergraduate}
	\imprimirfolhaderosto{}
	\imprimirfolhadeaprovacao
	\imprimiragradecimentos{text/1-pre-textuais/agradecimentos}
	\imprimirepigrafe{text/1-pre-textuais/epigrafe}
	\imprimirabstract{text/1-pre-textuais/abstract}
	\imprimirresumo{text/1-pre-textuais/resumo}
	\imprimirlistadeilustracoes
	\imprimirlistadetabelas
	%\imprimirlistadequadros
	%\imprimirlistadealgoritmos
	% \imprimirlistadecodigosfonte
	% \imprimirlistadeabreviaturasesiglas TODO: make this work !
	% \imprimirlistadesimbolos{text/1-pre-textuais/lista-de-simbolos}
	% \imprimirsumario
	% \tableofcontents
    \newpage

	\setcounter{table}{0}% Deixe este comando antes da primeira tabela.

	%Elementos textuais
	\chapter{Introdução}
\label{cap:introducao}

Séries temporais como uma categoria de dados ganhou tremenda importância no
advento de \emph{big data} por representar essencialmente qualquer tipo de
informação que evolui no tempo. Grande parte da teoria envolvida no campo
conhecido como análise de séries temporais é idêntica ou muito semelhante a
conceitos de sistemas lineares e processamento de sinais, mas a compreensão dos
temas na literatura científica e didática atual frequentemente não explora o
potencial analítico dessa interseção.
TODO: mention stochastic signal processing

O seguinte trabalho tem por objetivo abordar teoria clássica de séries
temporais por um ponto de vista de sinais e sistemas. Em uma extensa revisão de
literatura cada definição é introduzida notando paralelos em nomenclatura,
conceito e papel analítico entre os dois campos. Eventualmente conceitos
dificilmente capturados por quaisquer uma das áreas isoladamente são abordados
de forma integrada e a simplicidade resultante é notável. Em uma reflexão
teórica conclusiva o limite de comunicação dos campos é traçado. Por fim uma
aplicação prática usando o corpo teórico desenvolvido é apresentada, na qual
uma série de análises atípicas e informativas são realizadas por meio da
abordagem desenvolvida.

O capítulo 2 inicia com a apresentação de uma base de definições e
nomenclatura, sempre explorando os paralelos entre os campos abordados. Ao
final do capítulo o conceito crítico de estacionariedade é apresentado e
interpretado com devida atenção.

O capítulo 3 introduz teoria de decomposição, modelagem ARMA, modelagem sazonal
e contém a interpretação mais direta de séries temporais como saídas de
sistemas lineares. Essa interpretação é desenvolvida de forma a delimitar até
onde as duas áreas podem se comunicar de forma a produzir análises úteis.

O capítulo 4 discute a representação espectral de séries temporais univariadas.
São exploradas transformações lineares e não lineares. O desenvolvimento de uma
representação espectral para uma realização teoricamente infinita de um
processo estocástico é apresentada. Em seguida inicia-se uma discussão sobre
representações não lineares por meio de uma generalização natural da função de
autocorrelação e subsequentemente a classe de distribuições de Cohen.

O capítulo 5 brevemente discute testes estatísticos utilizados no capítulo 6.

O capítulo 6 apresenta um problema de extração de características para detecção
de anomalias em uma série temporal multivariada com sazonalidade elaborada. O
problema é abordado de forma prática - preocupações sobre sua resolução no
mundo real são levantadas e ferramentas de operacionalização de modelos são
utilizadas.

    \chapter*{Introdução}
\label{sec:teorica_classica_series_temporais_introducao}

\chapter{Definições e Propriedades}
\label{chap:teorica_classica_series_temporais_definicoes}

\section{Processo Estocástico}\label{sec:process}

Dado um conjunto arbitrário $\mathcal{T}$ um processo estocástico é uma família
${X(t,\omega)}$, $ t\in \mathcal{T}$ e $\omega \in \Omega$ de forma que para
cada $t \in T$, $\omega \in \Omega$ $X(t, \Omega)$ é uma variável aleatória. As
variáveis aleatórias podem ser reais ou complexas. Esse trabalho aborda apenas
processos estocásticos reais exceto quando explicitamente mencionado.

Supõe-se que a família de variáveis aleatórias seja definida em um mesmo
espaço de probabilidades $(\Omega, \mathcal{A}, P)$ com $\Omega$ representando
um espaço amostral, $\mathcal{A}$ uma $\sigma$-álgebra e $P$ uma medida de
probabilidade. Para propósitos desse trabalho podemos tomar o conjunto
$\mathcal{T}$ como $\mathbb{R}$, resultando em processos de tempo contínuo, e
$\mathbb{Z}$, resultando em processos de tempo discreto.

Para cada $t \in \mathcal{T}$ temos uma função de densidade de probabilidade
associada à variável aleatória $X(t_k, \omega)$ (assumindo que essa função exista).
Na prática um processo aleatório no mundo real é observado ao longo de $t$, tal que
$\omega$ seja fixado ao universo em que a observação ocorre. Sob essa condição
$X(t, \omega_k)$ é considerada uma realização do processo estocástico. Realizações também
são chamadas de \emph{sample record} em alguns livros de engenharia
e séries temporais na literatura estatística.

Para ilustrar os conceitos acima podemos pensar em um exemplo proposto por
~\cite{random_data} em que um gerador de ruído térmico é construído e sua tensão
ao longo de um intervalo do tempo é medida. Se um outro gerador fosse
construído sob condições e com propriedades idênticas sua tensão medida no
mesmo intervalo de tempo não seria idêntica, assim como a tensão medida sob
qualquer outro gerador idêntico. De fato cada registro de tensão é um exemplo
de infinitos registros que poderiam ter ocorrido. Nessa situação os registros
ou séries temporais de tensão são as realizações de um processo estocástico
representativo de todas as possíveis realizações.

O adequado estudo de séries temporais é consequência de um primeiro adequado
estudo sobre processos estocásticos, geradores dessas séries temporais. Essa
não é a intenção desse trabalho. Como em grande parte da teoria de séries
temporais estamos preocupados com o que podemos compreender ou inferir sobre
o processo estocástico gerador de uma realização a partir apenas de seu único
registro. Essa abordagem é essencial e de fato mais aplicável que um estudo
que se preocupa excessivamente com os processos geradores devido ao fato de dados
do mundo real frequentemente representarem realizações únicas. Não é possível
realizar novamente o índice Ibovespa entre 1970 e 2020 e muito menos ter acesso
à realização desses índices em universos paralelos.

Ao longo desse trabalho a distinção e referência ao processo gerador de uma série
temporal será feita quando necessário.


\section{Série Temporal}\label{ssec:definition}

Uma série temporal é um conjunto de observações realizadas sequencialmente no
tempo, indexadas de acordo com o momento em que foram observadas. As
observações representam a realização de um processo estocástico. Em alguns
contextos, como análise de processos industriai, a natureza do processo
subjacente é relevante para análise e modelagem de qualquer série temporal. Em
outros, como análise de séries financeiras, o sistema gerador das séries é tão
complexo que dificilmente conhecimento sobre sua dinâmica seja útil.

Assume-me, na linguagem de~~\cite{hamilton}, que um conjunto de amostras
$\mathbf{y}_t = (y_1, y_2, y_3 \dots y_T)$ pode ser interpretado como um
segmento finito de uma sequência duplamente infinita:

$${\mathbf{y}}_{t=-\infty}^{\infty} = ({\dots, y_{-1},y_0, \overbrace{y_1, y_2, y_3, \dots, y_T}^{\text{Série Observada}}, y_{T+1}, y_{T+2}}\dots)$$

\vspace{1cm}

Apesar de parecer pouco tangível, de fato qualquer série observada é
satisfatoriamente descrita dessa forma. Em um contexto industrial, por exemplo,
o valor de uma variável de processo é zero até a planta ser construída e entrar
em operação, assume valores representativos ou não da dinâmica de interesse
(assumindo valores irrelevantes quando a planta não está em operação plena), e
tendendo ao infinito retorna a zero quando a planta for desativada.

Essa interpretação de uma série temporal é importante ao implicitamente
insinuar que o processo existe em um intervalo temporal mais abrangente do que
o observado. É portanto necessário se questionar sobre quanto os
dados representam o processo analisado e em quais intervalos de tempo.

Séries temporais são inerentemente diferentes de dados tabulares por
representarem amostras de um mesmo processo estocástico ao invés de amostras
aleatórias de uma população. Não faz sequer sentido discutir uma população no
contexto de séries temporais uma vez que estamos restritos à realizações
observadas de um processo estocástico. Essa relevante diferença impede
propriedades estatísticas agradáveis consequentes da independência amostral
associada à dados tabulares apropriadamente amostrados como a lei do grandes
números e o teorema do limite central.

A forma mais natural de analisar uma série temporal é visualizar seus valores
no tempo, como ilustra a imagem~\ref{fig:example}.

\begin{figure}[H]
    \centering
    \includegraphics[scale=0.6]{figures/white_noise.png}
    \caption{Visualização no tempo de ruído branco}
    \label{fig:example}
\end{figure}


\section{Operador de Atraso(\emph{Lag})}

É importante introduzir o operador de atraso ou \emph{lag}.

Dadas as séries $\mathbf{y}_t = (y_1, y_2, y_3 \dots y_T)$ e
$\mathbf{x}_t = (x_0, x_1, x_2 \dots x_{T-1})$ tal que

$$ \mathbf{y}_t = \mathbf{x}_{t-1}$$

isso é,

$$ y_1 = x_0 $$
$$ y_2 = x_1 $$
$$ \vdots $$
$$ y_T = x_{T-1} $$

Podemos definir $\mathbf{x}_t$ em função de $\mathbf{y}_t$ como:

$$ \mathbf{x}_t = L\mathbf{y}_t $$

tal que

$$\mathbf{y}_{t-1} = L\mathbf{y}_t$$

Observamos que o operador de atraso atrasa uma série temporal em uma unidade de
tempo. Uma breve divagação matemática~~\cite{hamilton} permite definir o
operador com propriedades muito semelhantes às de multiplicação dos números
reais, como associatividade, comutatividade e distribuição. Para atrasar
múltiplas unidades de tempo temos que:

\vspace{1cm}

$$L(L(\mathbf{y_t})) = L(\mathbf{y}_{t-1}) = \mathbf{y}_{t-2} = L^2 \mathbf{y}_{t}$$

\vspace{1cm}

de forma que

$$ L^n {\mathbf{y}} =  \mathbf{y}_{t-n}$$

\vspace{1cm}

Um uso importante do operador, decorrente de suas propriedades algébricas, é
exemplificado na seguinte expansão

$$ (aL^2 + bL^3) \mathbf{y}_{t} =  a\mathbf{y}_{t-2} + b\mathbf{y}_{t-3}$$

Conhecimento do operador de atraso é importante para compreender a literatura
de séries temporais e facilita comunicação objetiva de análises cotidianas.
O presente trabalho usa do operador para descrever uma série de modelos.

Como nota final é importante mencionar que alguns livros~~\cite{chatfield}
~~\cite{stoffer} usam a letra $B$ para denotar o operador de atraso e que na
maior parte dos recursos \emph{online} o operador é referido por seu nome em
inglês, \emph{lag}.

A analogia entre o operador de atraso e a variável complexa $e^{-j\omega} = z^{-1}$
é clara, com a relevante diferença que $z^{-1}$ atrasa um sinal em uma unidade
de tempo se a operação for realizada no domínio $z$ enquanto o operador de
atraso atua diretamente no domínio do tempo. Essa característica permite que
filtros com equações de recorrência complexas sejam representados de forma
compacta no domínio do tempo por meio de polinômios de atraso.

Uma propriedade interessante decorrente da equivalência entre $z^{-1}$ e $L$ é
que pode se pensar em um plano $L$ cuja análise é igualmente informativa à do
plano z, notando que o espaço é de certa forma invertido. Uma análise da
posição dos polos de um sistema representado por meio de um polinômio em $L$
conclui que o sistema é instável se tais polos estiverem fora do círculo unitário,
contrário do que conhecemos do plano z.

Por fim vale mencionar que alguns autores como~\cite{aguirre} usam a notação $q^{-1}$
para esse operador.

\section{Operador de Diferença}\label{sec:diff}

O operador de diferenças $\nabla$ ou  $\Delta$ é o equivalente discreto da
operação contínua de derivação e opera sob uma série temporal
$\mathbf{y}_t$ da seguinte forma:

$$ \nabla \mathbf{y}_t = (1 - L)\mathbf{y}_t = \mathbf{y}_t - L\mathbf{y}_t = \mathbf{y}_t - \mathbf{y}_{t-1} $$

O operador possui propriedades de associatividade e distribuição, tal que

$$ \nabla^2 \mathbf{y}_t = \nabla(\nabla(\mathbf{y}_t)) = \nabla(\mathbf{y}_t - \mathbf{y}_{t-1}) = \nabla \mathbf{y}_t - \nabla \mathbf{y}_{t-1} = \mathbf{y}_t - 2 \mathbf{y}_{t-1} + \mathbf{y}_{t-2} $$

\section{Tendência(\emph{Trend})}

A variação do valor esperado de um processo estocástico é denominado tendência.
A partir de uma série temporal definimos tendência como a variação de sua média
amostral. A imagem ~\ref{fig:trend} ilustra uma série com tendência linear.
Observa-se que ao longo do tempo a média das observações cresce linearmente.
Tendências de séries reais frequentemente seguem um perfil
logarítmico~~\cite{chatfield}, como na figura~\ref{fig:log_trend}. Nesse caso
uma transformação exponencial da série, isso é, a aplicação de uma função
exponencial a cada observação, tornaria a tendência linear.

\begin{figure}[H]
    \centering
    \includegraphics[scale=0.6]{figures/trend.png}
    \caption{Série temporal com tendência linear}
    \label{fig:trend}
\end{figure}

\begin{figure}[H]
    \centering
    \includegraphics[scale=0.6]{figures/log_trend.png}
    \caption{Série temporal com tendência logarítmica}
    \label{fig:log_trend}
\end{figure}

\section{Sazonalidade}\label{sec:seasonality}

A variação periódica de média móvel das observações de uma série temporal é
denominada sazonalidade. Em séries no contexto de finanças sazonalidade
frequentemente segue ciclos de calendário como anual, mensal, semestral, etc.
No contexto mais amplo de séries temporais sazonalidade apresenta período
arbitrário, apesar da linguagem em torno dessa propriedade estar muito
associada aos períodos anteriormente mencionados.

Um exemplo de série temporal com sazonalidade é ilustrado na
figura~\ref{fig:seasonality}.

\begin{figure}[H]
    \centering
    \includegraphics[scale=0.6]{figures/seasonality.png}
    \caption{Série temporal com sazonalidade senoidal de período arbitrário}
    \label{fig:seasonality}
\end{figure}

Na presença de tendência sazonalidade pode ser considerada aditiva, se sua
variação for constante em torno da tendência, e multiplicativa, se sua variação
depender o valor da tendência. Exemplos de sazonalidade aditiva e
multiplicativa são dados pelas figuras~\ref{fig:add_seasonality} e
~\ref{fig:mult_seasonality}, respectivamente. Discernir entre os dois tipos de
sazonalidade é importante para modelagem.

\begin{figure}[H]
    \centering
    \includegraphics[scale=0.6]{figures/add_seasonality.png}
    \caption{Série temporal com sazonalidade aditiva}
    \label{fig:add_seasonality}
\end{figure}

\begin{figure}[H]
    \centering
    \includegraphics[scale=0.6]{figures/mult_seasonality.png}
    \caption{Série temporal com sazonalidade multiplicativa}
    \label{fig:mult_seasonality}
\end{figure}


\section{Autocorrelação}

TODO: later analogy with convolution would be cool
TODO: list properties of autocorrelation function !

A função de autocorrelação é definida para processos estocásticos como a
correlação de Pearson entre valores do processo em instantes de tempo
diferentes. A função de autocovariância entre os instantes de tempo $t_1$ e
$t_2$ é dada pela seguinte equação

\begin{equation}\label{eq:raw_autocorr}
    gamma_{xx}(t_1, t_2) = E[(X_{t_1} - \mu_{t_1})(X_{t_2} -\mu_{t_2})]
\end{equation}

Normalizando a autocovariância obtemos a autocorrelação

$$\rho_{xx}(t_1, t_2) =\frac{K_{xx}(t_1, t_2)}{\sigma_{t_1}\sigma_{t_2}}$$

Para processos estacionários (seção~\ref{sec:stationarity}) a autocovariância,
e consequentemente a autocorrelação, é função apenas do atraso $\tau = |t_1 -
t_2|$. Temos então que

$$\rho_{xx}(\tau) =\frac{K_{xx}(\tau)}{\sigma_{t_1}\sigma_{t_2}}$$

Para uma série temporal, isso é, uma única  realização de um processo
estocástico, a função de autocorrelação estacionária (tipicamente chamada
apenas de função de autocorrelação) pode ser definida diretamente a partir da
definição de correlação amostral sob as seguintes premissas

\begin{enumerate}
    \item O processo estocástico gerador da série temporal é estacionário
    \item O número de observações  $N$ de $\mathbf{y}_t$ é suficientemente
    grande ($N \approx 100$)
\end{enumerate}

resultando na equação ~\ref{eq:autocorr}, onde o subscrito duplo é omitido.
Note que o atraso é discreto, indicado por $k$.

\vspace{1cm}

\begin{equation}\label{eq:autocorr}
    \rho_y(k) = \frac{\sum_{t=1}^{N-k}(y_t - \bar{y})(y_{t+k}-\bar{y})}{\sum_{t=1}^{N}(y_t - \bar{y})^2}  , \hspace{1cm} k = 0, 1, 2, \dots
\end{equation}

Alguns comentários sobre a nomenclatura da função são apropriados. A literatura
de engenharia tende a usar os termos autocovariância e autocorrelação de forma
intercambiável para designar a definição de autocovariância apresentada. A
literatura estatística assume as definições abordadas acima, que serão usada no
restante desse trabalho.

Além disso, a função de autocorrelação como apresentada pela
equação~\ref{eq:raw_autocorr} é definida para um processo estocástico não
necessariamente estacionário, apesar do termo ser usado para descrever a
equação~\ref{eq:autocorr}. A aplicação da equação que assume estacionariedade
em uma série não estacionária resulta em correlações informativas praticamente
apenas disso. Uma modificação estratégica da função dada
por~\ref{eq:raw_autocorr} resulta na chamada função de autocorrelação
instantânea (seção~\ref{ssec:inst_autocorr}), que é usada para representar
séries temporais não estacionárias.

\vspace{1cm}

\subsection{Autocorrelação Parcial}\label{ssec:partial_acorr}

É interessante mencionar a existência de autocorrelação parcial nessa seção
juntamente de uma descrição em alto nível do que esse valor representa. Sua
definição formal será apresentada na seção~\ref{ssec:AR(p)}.

Autocorrelação parcial foi introduzida por Box e Jenkins em~\cite{box} como uma
ferramenta auxiliar na identificação de modelos. O valor $\phi_{kk}$ é
definido como a correlação parcial entre $\mathbf{y}_t$ e $\mathbf{y}_{t - k}$,
isso é, a correlação restante entre $\mathbf{y}_t$ e $\mathbf{y}_{t - k}$ após
levar em consideração a contribuição de $\mathbf{y}_t$, $\mathbf{y}_{t - 1}$
$...$ $\mathbf{y}_{t - k + 1}$.

\subsection{Correlalograma}\label{ssec:correlalogram}

Um correlalograma é um gráfico de barras representativo da autocorrelação ou
autocovariância em $k$ amostras de uma série temporal ($k=0, 1, 2, \dots$), de
forma que a primeira barra represente a autocorrelação entre $y_t$ e si mesmo
(sempre igual à 1), a segunda entre $y_t$ e $y_{t-1}$, a terceira entre $y_t$ e
$y_{t-2}$, e assim por diante. O correlalograma da série visualizada pela
figura~\ref{fig:trend} é ilustrado na figura~\ref{fig:correlalogram}.

\begin{figure}
    \centering
    \includegraphics[scale=0.5]{figures/corr_trend.png}
    \caption{Visualização de correlalograma de série com tendência linear.
    Observe que as autocorrelações decaem lentamente ao decorrer dos atrasos,
    comportamento típico de tendências determinísticas.}
    \label{fig:correlalogram}
\end{figure}

\begin{figure}
    \centering
    \includegraphics[scale=0.5]{figures/corr_seasonality.png}
    \caption{Visualização de correlalograma de série com sazonalidade.
    Observe que o padrão senoidal da série é reproduzido nas autocorrelações.}
    \label{fig:corr_season}
\end{figure}

A figura~\ref{fig:correlalogram} informa um intervalo de relevância dado por um
sombreamento vermelho. Qualquer valor de autocorrelação dentro desse intervalo
é estatisticamente insignificante e pode ser considerado igual a zero.

O correlalograma é uma ferramenta indispensável em análise de séries temporais
para tarefas como detecção de estacionariedade, identificação de sazonalidade,
análise de resíduo, engenharia de características, escolha de modelo e
identificação de ruído branco(seção~\ref{sec:white_noise}), entre outros. No
contexto de identificação de sistemas autocovariância e autocorrelação e
portanto o correlalograma desempenham um importante papel na identificação de
propriedades de sinais e sistemas imersos em ruído devido à robustez ao ruído
da operação de correlação cruzada~\cite{aguirre}. A figura~\ref{fig:noisy_sine}
apresenta um sinal imerso em ruído cuja natureza periódica subjacente se torna
mais visível por meio de seu correlalograma.

\begin{figure}
    \centering
    \includegraphics[scale=0.5]{figures/noisy_seasonal_signal.png}
    \caption{Sinal periódico imerso em ruído e sua correspondente autocorrelação.}
    \label{fig:noisy_sine}
\end{figure}

É importante mencionar que o correlalograma de uma série com tendência
determinística, como a da figura~\ref{fig:correlalogram}, apresenta o
comportamento observado de autocorrelações altas com pouca atenuação ao longo
dos atrasos. De forma análoga o correlalograma de uma série com sazonalidade
apresenta periodicidade que reproduz seu padrão temporal, como ilustra a
figura~\ref{fig:corr_season}, correlalograma da série da
figura~\ref{fig:seasonality}. O primeiro correlalograma é informativo até
certo ponto: informa simplesmente que a série apresenta tendência. Para
analisar tais séries de forma mais produtiva, a fim de elaborar um possível
modelo, por exemplo, é importante que a série seja estacionária
(seção~\ref{sec:stationarity}). É inclusive afirmado em alguns textos da
literatura estatística~\cite{chatfield}, que um correlalograma só faz que
sentido se a série associada for estacionária, observação mais geral e rigorosa
das restrições de uso da equação~\ref{eq:autocorr}.

Na engenharia a análise do correlalograma de sinais não estacionários é
utilizada para investigação da adequação de tempo de amostragem, onde uma
autocovariância com valores lentamente decrescentes e um mínimo local indica que
o sinal pode estar superamostrado, propriedade indesejável que pode resultar em
problemas computacionais além de desperdício de memória.

TODO: add sampling analysis of silica series? later maybe

\section{Estacionariedade}\label{sec:stationarity}

Um processo estocástico $\mathbf{X}(t)$ é considerado estacionário no
sentido amplo se atender às seguintes três condições:

\begin{enumerate}
    \item $E(\mathbf{X}(t)) = \mu$
    \item $Var(\mathbf{X}(t)) = \sigma^2$
    \item $Cov[\mathbf{X}(t), \mathbf{X}(t+\tau)] = \gamma(\tau)$
\end{enumerate}\vspace{.5cm}

e estacionário no sentido restrito se sua distribuição de probabilidade $P(x)$
for idêntica para todos os instantes de tempo, isso é, $P(x)_{t_{i}} = P(x)_{t_{j}}$
$\forall i, j \in \mathcal{T}$. Essa restrição é frequentemente comunicada como
a necessidade de todos os momentos da distribuição $P(x)$ serem idênticos e
invariantes ao tempo. Nesse trabalho, assim como em grande parte da literatura
estatística e de engenharia, o termo estacionário se refere a estacionariedade
no sentido amplo. Isso é parcialmente justificado pelo fato de verificação
de estacionariedade no sentido amplo tipicamente ser condição suficiente para
assumir estacionariedade no sentido restrito, segundo~\cite{random_data}.

A definição de estacionariedade apresentada descreve a família de sinais
representada por um processo estocástico. Uma definição de estacionariedade
para realizações únicas de um processo estocástico, isso é, séries temporais,
demanda o estabelecimento de resquisitos amostrais. Podemos dizer que uma
série temporal é estacionária se suas propriedades amostrais de esperança,
variância e covariância sejam apropriadamente invariantes de forma que
flutuações em seu valor sob diferentes janelas de tempo sejam explicáveis por
variações resultantes de amostragem. Sob essa perspectiva podemos traduzir
os requisitos de estacionariedade para uma série $\mathbf{x}_t$ como a seguir

\begin{enumerate}
    \item A média da série $\mathbf{x}_t$ é constante ao longo do tempo
    \item A variância da série $\mathbf{x}_t$ é constante ao longo do tempo
    \item A autocorrelação de $\mathbf{x}_t$ depende apenas do atraso
\end{enumerate}\vspace{.5cm}

\begin{figure}[h]
    \centering
    \includegraphics[scale=0.6]{figures/stationarity_examples.png}
    \caption{Conjunto de séries demonstrando diferentes níveis de
    estacionariedade.}
    \label{fig:stationarity}
\end{figure}

Um bom exemplo de graus de estacionariedade em séries temporais dado por
Athanasopoulos e Hyndman~~\cite{athana} é ilustrado pela
figura~\ref{fig:stationarity}. As séries das figures~\ref{fig:stationarity}.a,
~\ref{fig:stationarity}.e e~\ref{fig:stationarity}.i demonstram clara
tendência, sendo portanto não estacionárias. As séries das figures
~\ref{fig:stationarity}.d, ~\ref{fig:stationarity}.h, ~\ref{fig:stationarity}.i
são igualmente não estacionárias por apresentarem clara sazonalidade, enquanto
a série da figura~\ref{fig:stationarity}.g aparenta ser sazonal mas apresenta
picos aperiódicos de intensidade muito distinta, sendo considerada estacionária
pelos autores. O caso da série da figura~\ref{fig:stationarity}.g é um
complicado por aparentar também violar o requisito de variância constante. Para
concluir assim como os autores que a série é estacionária é necessário mais do
que uma investigação visual; o texto que discute a série leva em conta seu
processo gerador~~\cite{athana}.

As séries das figures~\ref{fig:stationarity}.c e ~\ref{fig:stationarity}.f são
aparentemente não estacionárias no intervalo observado por demonstrarem
variação em sua média móvel mas podem ser um caso de raízes unitárias.

Por eliminação temos que apenas as séries ilustradas pelas
figures~\ref{fig:stationarity}.b e ~\ref{fig:stationarity}.g são estacionárias,
o caso de ~\ref{fig:stationarity}.b contendo uma clara anomalia.

O exemplo de análise de estacionariedade da figura~\ref{fig:stationarity}
demonstra a imprecisão da abordagem visual para essa tarefa. Torna-se
necessário o estabelecimento de procedimentos mais objetivos para detecção de
estacionariedade.

\subsection{Importância de Estacionariedade}

Estacionariedade é uma propriedade desejável de se observar em uma série
temporal para fins analíticos e de modelagem.

Há algumas formas de compreender como estacionariedade colabora para a
modelagem bem sucedida de uma série temporal.

Podemos pensar na propriedade de estacionariedade como um tipo de estrutura de
dependência. Se as amostras $X_1, X_2, \dots, X_N$ de um conjunto $\mathbf{X}$
forem independentes entre si temos formas interessantes de modelar a função
geradora de $\mathbf{X}$ como o teorema do limite central, lei dos grandes
números, etc. Há apenas uma forma de um conjunto amostral ser independente mas
muitas formas de ser dependente, tornando difícil o estabelecimento de recursos
eficientes para modelagem geral de processes dependentes. Séries temporais
sendo naturalmente observações de processos dependentes
(seção~\ref{ssec:definition}) é interessante definir estruturas de dependência
que permitam o uso de tais recursos. Estacionariedade é uma estrutura de
dependência que permite aplicar propriedades úteis de independência em séries
temporais. Abordando a mesma ideia mais intuitivamente podemos pensar no
seguinte exemplo: se um processo possui valor esperado, variância constante e
autocorrelação invariante ao tempo podemos por meio da lei dos grandes números
estimar seu valor esperado e variância com cada vez mais confiança a partir da
média e variância amostral, respectivamente. O mesmo argumento intuitivo se
estende
analogamente para o caso de aprendizado de máquina, no qual por meio de teoria
de aprendizado estatístico é possível argumentar que uma série estacionária é
``mais fácil'' de aprender.

De forma mais quantitativa o teorema de decomposição de Wold~~\cite{chatfield}
conclui que qualquer série temporal estacionária pode ser representada pela
seguinte combinação linear

$$\mathbf{y}_t = \sum_{j=0}^\infty b_j Z_{t-j} + \eta_t$$

No qual $\eta$ representa uma série determinística e $Z_t$ um processo
puramente aleatório (seção~\ref{sec:white_noise}). O leitor reconhecerá parte
da expressão acima como um processo $MA(\infty)$ (seção~\ref{sssec:MA(p)}).
Esse resultado tem como consequência a importante conclusão que qualquer série
estacionária é possivelmente aproximável por um modelo MA e portanto, via
invertibilidade, modelos AR e ARMA (seção~\ref{ssec:stability_invertibility}).

Por fim estacionariedade permite o uso de uma série de modelos que serão
discutidos na seção sobre modelos estacionários. Esses métodos são bem
compreendidos e implementados, facilitando sua interpretação, uso e
sustentação.

\subsection{Categorias Básicas de Não Estacionariedade}\label{ssec:taxonomy}

Como extensão do argumento sobre estruturas de dependência na seção anterior
podemos afirmar que, sendo estacionariedade um padrão de dependência, temos
infinitas formas de não estacionariedade, retornando ao caso de dependência
generalizada. É interessante identificar nesse universo de dependência padrões
de séries não estacionárias que são facilmente transformadas em séries
estacionárias.

Uma série temporal com presença de tendência determinística, como ilustrada na
figura~\ref{fig:trend}, pode ser representada pela seguinte expressão:

$$  y_t = e_t + f(t) + \varepsilon_t  \hspace{1cm}\text{onde} \hspace{.4cm}\varepsilon_t \sim \hspace{.2cm}\text{i.i.d.} \hspace{.2cm}\mathcal{N}(0, \sigma^2)$$

Na qual $e_t$ representa uma série estacionária, $f(t)$ uma função
determinística do tempo e $\varepsilon_t$ ruído
branco(seção~\ref{sec:white_noise}). Nota-se que $f(t)$ é uma função
monotônica arbitrária tal que $y_t$ seja uma série não estacionária. No caso da
figura~\ref{fig:trend} temos $f(t)$ linear e na figura ~\ref{fig:log_trend}
logarítmica. Uma série temporal demonstrando esse tipo de não estacionariedade
é considerada \textbf{tendência-estacionária}, uma vez que simplesmente
removendo a tendência $f(t)$ temos estacionariedade. Isso pode ser feito de
várias formas, talvez com maior simplicidade diferenciando a série. Métodos
mais sofisticados incluem decomposição ETS (seção~\ref{sec:decomposition}) e
regressão com finalidade de modelar $f(t)$ de forma que o resíduo represente
uma a série estacionária $e_t + \varepsilon_t$.

\begin{figure}[h]
    \centering
    \includegraphics[scale=0.5]{figures/random_walk.png}
    \caption{Visualização de caminhada aleatória com $y_0=5$}
    \label{fig:random_walk}
\end{figure}

Uma série com presença de tendência estocástica pode ser classificada de
maneira semelhante. O exemplo mais simples de tal série é gerada por um
passeio aleatório, definido pelo seguinte processo, visualizado pela figura
~\ref{fig:random_walk}:

$$  y_t = y_{t-1} + \varepsilon_t  \hspace{1cm}\text{onde} \hspace{.4cm}\varepsilon_t \sim \hspace{.2cm}\text{i.i.d.} \hspace{.2cm}\mathcal{N}(0, \sigma^2)$$

Por meio de um desenvolvimento recursivo do processo podemos escrever:

$$ y_t = (y_{t-2} + \varepsilon_{t-1}) + \varepsilon_{t} $$
$$ y_t = ((y_{t-3} + \varepsilon_{t-2}) + \varepsilon_{t-1}) + \varepsilon_{t} $$
$$ \vdots $$
$$ y_t = \sum_{j=0}^{N-1} \varepsilon_{t-j} + y_0$$
\vspace{1cm}

Resultado a partir do qual a não estacionariedade de $y_t$ se torna evidente,
uma vez que

$$ var(y_t) = \sigma^2 t $$

Além da covariância ser dependente do tempo.

Uma forma simples de tornar a série estacionária é diferenciá-la em primeira
ordem, isso é, aplicar o operador de diferença primeira:

$$ \nabla y_t = y_t - y_{t-1} $$
$$ y_t - y_{t-1} = \varepsilon_t$$
$$ \nabla y_t =  \varepsilon_t$$

Sabemos pela seção~\ref{sec:white_noise} que ruído branco é um processo
estacionário.

A caminhada aleatória é denominada uma série \textbf{diferença-estacionária}
pelo fato da operação de diferença introduzir estacionariedade. Essa é uma
forma tão comum de não estacionariedade que a ideia de ``diferenciar uma
série antes de fazer qualquer coisa'' é proeminente entre profissionais de
dados, apesar de que geralmente necessita-se apenas de estacionariedade
Essa prática é parcialmente justificada considerando que a maior
parte das séries temporais ``reais'' são não estacionárias e frequentemente
diferencialmente estacionárias.

É igualmente possível que uma série diferença-estacionária seja estacionária em
sua $n$-ésima diferença, tal que estacionariedade seja observada por uma
operação de diferenças de ordem $n$. A ideia de tirar sucessivas diferenças
até atingir estacionariedade é fundamental no método de Box-Jenkins, por
exemplo.

Séries diferença-estacionárias apresentam raízes unitárias e os dois termos são
frequentemente usados nos mesmos contextos.

Podemos resumir as definições das categorias de não estacionariedade abordadas
nessa seção assim como suas implicações como segue:

\begin{enumerate}
    \item \textbf{Estacionariedade em Tendência}: Uma série é considerada
        tendência-estacionária se apresentar uma tendência determinística. No
        caso de anomalias ou eventos de perturbação séries com esse tipo de
        tendência retornam ao valor da tendência ao longo do tempo,
        ``esquecendo'' o evento perturbador. Esse tipo de série se torna
        estacionária pela remoção da tendência determinística, processo
        realizado por meio de regressão da tendência, por diferenciação, por
        decomposição, etc.
    \item \textbf{Estacionariedade Diferenciável}: Uma série é considerada
        diferença-estacionária se apresentar uma tendência estocástica. No
        caso de anomalias ou eventos de perturbação séries com esse tipo de
        tendência são irreversivelmente afetadas,
        ``lembrando'' do evento perturbador. Esse tipo de série se torna
        estacionária por diferenciação em ordem $n$. Possui raízes unitárias
        e é frequentemente discutida nessa linguagem.

\end{enumerate}

\section{Ruído Branco}\label{sec:white_noise}

TODO: inserir propriedades de ruído branco

Uma série temporal $\mathbf{x}_t$ gerada por um processo $\mathbf{X}(t)$ é
considerada ruído branco ou um processo puramente aleatório se atender às
seguintes três condições:

\begin{enumerate}
    \item $E(\mathbf{X}(t)) = 0$
    \item $Var(\mathbf{X}(t)) = \sigma^2$
    \item $Cov[\mathbf{X}(t), \mathbf{X}(t+\tau)] = 0$
\end{enumerate}\vspace{.5cm}

Que podem ser interpretadas da seguinte forma

\begin{enumerate}
    \item A média da série $\mathbf{x}_t$ é nula ao longo do tempo
    \item A variância da série $\mathbf{x}_t$ é constante ao longo do tempo
    \item Não há correlação entre as amostras de $\mathbf{x}_t$
\end{enumerate}\vspace{.5cm}

Observa-se que ruído branco é um caso específico de estacionariedade, se
diferenciando pela especificação do valor esperado e autocorrelação entre
quaisquer amostras em zero. A compreensão da definição e capacidade de
identificação de ruído branco é importante para análise de resíduos.

	\chapter{Teoria Univariada}
\label{chap:univariate_theory}

\section*{Introdução}

O seguinte capítulo tem por objetivo introduzir uma parte pequena mas
representativa do corpo clássico de análise de séries temporais univariadas
usando uma base de vocabulário e terminologia desenvolvida no capítulo 2.

\newpage

\section{Decomposição de Séries Temporais}
\label{sec:decomposition}

Decomposição em séries temporais tipicamente descreve o processo de
representação de uma série por uma combinação linear de três componentes:
tendência, sazonalidade e resíduos. A decomposição de uma série $y_t$ pode ser
aditiva ou multiplicativa como expressada pelas equações~\ref{eq:add_decomp} e
~\ref{eq:mult_decomp} respectivamente.

\begin{equation}\label{eq:add_decomp}
    y_t = S_t + T_t + R_t
\end{equation}

\begin{equation}\label{eq:mult_decomp}
    y_t = S_t \cdot T_t \cdot R_t
\end{equation}

A escolha de decomposição aditiva ou multiplicativa deve ser feita de acordo
com o tipo de sazonalidade (seção~\ref{sec:seasonality}).

Como herança de econometria grande parte dos algoritmos clássicos de
decomposição (X11~\cite{x11}, SEATS~\cite[capítulo~5.2]{SEATS} e derivados) são
baseados em período de sazonalidade anuais, semestrais, trimestrais e
mensais~\cite{athana}. A incapacidade desses algoritmos de processar dados com
período menor os torna pouco utilizáveis no contexto de sinais elétricos, por
exemplo, cujas séries tipicamente são de período inferior a um dia,
apresentando padrões sazonais com período semelhante.

Antes de prosseguir é interessante comentar sobre os principais objetivos de
decomposição:

\begin{enumerate}

    \item Análise: O processo de pensar sobre, elaborar, ajustar e observar
    resultados de decomposição são altamente informativos da natureza da
    série sob análise. A operação é frequentemente usada para fins de análise
    exploratória.

    \item Indução de estacionariedade: Como tendência sempre é e sazonalidade
    pode ser um padrão cuja presença qualifica não estacionariedade sua
    remoção pode tornar uma série ``mais estacionária" ou pelo menos mais
    apropriada para modelagem por meio de modelos não estacionários.

    \item Detecção de Anomalias: Eventos anômalos são tipicamente dissociados
    de e ocultados por estruturas de tendência e sazonalidade. A remoção
    dessas estruturas tende a expor anomalias de forma mais detectável.

    \item Modelagem e Previsão: A representação de uma série por meio de três
    componentes distintas é interessante para problemas de previsão pela
    possibilidade de desenvolver modelos e previsões para cada componente de
    acordo com suas propriedades. Os modelos independentes tendem a ser menos
    complexos, mais generalizáveis e mais robustos do que um modelo adequado
    para a série original, se existir.

\end{enumerate}

\subsection{Modelagem de Tendência}

Uma operação essencial em decomposição de séries temporais é a determinação
de uma tendência subjacente dos dados. Os principais procedimentos para tal
são abordados a seguir.

\subsubsection{Média Móvel}\label{sssec:MA(p)}

Um filtro de média móvel de ordem $m$ é tipicamente expressado como:

\begin{equation}\label{eq:ma}
    MA_{t_{m}} = \frac{1}{m} \sum_{i=-k}^{i=k} y_{t+i}
\end{equation}

A operação é visualizada pela figura~\ref{fig:MA}.

\begin{figure}[H]
    \centering
    \includegraphics[scale=0.6]{figures/moving_average.png}
    \caption{Visualização de aplicação de filtro média móvel com m=15}
    \label{fig:MA}
\end{figure}


\subsubsection{Regressão Localizada/Filtro de Savitzky-Golay}

Uma forma mais robusta e granular de modelagem de tendência é o algoritmo
LOESS, baseado em regressão localizada e conhecido em linguagem de processamento
de sinais como o filtro de Savitzky-Golay. Apesar do algoritmo ser simples
seu desenvolvimento será omitido por ser matematicamente verboso. É importante
que o leitor conheça a existência do método LOESS, sabendo que ele oferece uma
alternativa mais precisa para modelagem de tendência do que média móvel,
apesar de ser potencialmente mais computacionalmente complexo. Uma breve
intuição sobre seu funcionamento é apresentada a seguir.

A ideia fundamental do algoritmo é repartir a série temporal em grupos de
pontos menores, calculando uma regressão (tipicamente mas não necessariamente
linear) nesses grupos menores de forma a construir uma curva linear por partes
que aproxima a tendência da curva original. O tamanho dos grupos é arbitrário
e tipicamente informado como uma fração do tamanho da série completa e os grupos
em si são determinados pelos $n$ pontos mais próximos a um elemento chamado de
``ponto focal'' que é incrementado a cada iteração. O valor dos incrementos dos
pontos focais também é um parâmetro.

A figura~\ref{fig:loess} ilustra o resultado do algoritmo LOESS para a mesma
série analisada pela figura~\ref{fig:MA}. A figura ilustra curvas de tendência
modeladas via LOESS para diferentes tamanhos de grupos. Os tamanhos são
informados como porcentagem do comprimento total do sinal. Observa-se que o uso
de uma porcentagem menor do sinal aumenta a localidade da regressão tornando o
ajuste da tendência mais sinuosa.

\begin{figure}[H]
    \centering
    \includegraphics[scale=0.6]{figures/loess.png}
    \caption{Visualização de algoritmo LOESS}
    \label{fig:loess}
\end{figure}

\subsection{Filtro de Kalman}

TODO: talk (and research about !) the use of kalman filter for moving average

\subsection{Modelagem de Sazonalidade(Ajuste de Sazonalidade)}

A operação fundamental de decomposição é a modelagem ou ajuste de sazonalidade.
Esse processo se preocupa em identificar estruturas periódicas em uma série
temporal. Modelagem de sazonalidade em séries temporais é de fato uma área em
si~\cite{seasonal_modelling}. A seção~\ref{sec:seasonality_models} é dedicada
à uma pequena exploração desses métodos. Abordamos a seguir sua versão mais
simples, tipicamente implementada no processo de decomposição de séries
temporais.

Uma possibilidade para estimativa da componente sazonal é determinar o ``padrão
sazonal médio'' da série sem tendência. A série de tamanho $n$ sem tendência é
divida em $s$ segmentos de $m$ amostras, no qual $s = \frac{n}{m}$. Usando
livremente a vírgula como símbolo para concatenação podemos enumerar os
segmentos como $$\chi_1, \chi_2, ... \chi_s$$ onde um segmento arbitrário
$$\chi_i = \chi_{i_{1}}, \chi_{i_{1}} ... \chi_{i_{m}}$$

O padrão sazonal modelado $S'_t$ corresponde ao segmento médio, isso é,

$$ S'_t = \left(\frac{1}{s} \sum_{k=1}^{k=s} \chi_{1_{k}}\right), \left(\frac{1}{s} \sum_{k=1}^{k=s} \chi_{2_{k}}\right), \left(\frac{1}{s} \sum_{k=1}^{k=s} \chi_{3_{k}}\right) \hdots \left(\frac{1}{s} \sum_{k=1}^{k=s} \chi_{m_{k}}\right)$$

A componente sazonal $S_t$ é então dada pela concatenação de $m$ cópias de
$S'_t$

$$ S_t = \overbrace{S'_t, S'_t, \hdots, S'_t}^{s\text{ vezes}} $$

\subsection{Decomposição Clássica (Aditiva)}\label{ssec:classical_decomposition}

O algoritmo de decomposição clássica aditiva é simples e capaz de identificar
períodos arbitrários de sazonalidade. O procedimento é apresentado passo a
passo.

\subsubsection{Passo 1}

\textbf{Identificar o período de sazonalidade que se deseja modelar.}

O período $m$ de sazonalidade corresponde à duração de um ciclo de um padrão
periódico observável nos dados. A determinação da duração desse ciclo não é
necessariamente trivial, especialmente tratando de séries de alta frequência,
sendo interessante inclusive usar métodos automáticos para determinação da
distância entre amostras iniciais dos períodos de sazonalidade. Uma
possibilidade, por exemplo,  é analisar os picos da função de autocorrelação e
acusar a distância entre os picos como um período amostral.

\subsubsection{Passo 2}

\textbf{Modelar tendência }$\mathbf{T_t}$

A tendência $T_t$ é modelada por meio de algum dos métodos mencionados, por
exemplo, um filtro média móvel com tamanho de janela $w$.

\subsubsection{Passo 3}

\textbf{Remover tendência da série}

A componente sem tendência é dada por $y_t - T_t$. A série resultante deve
possuir média aproximadamente nula.

\subsubsection{Passo 4}

\textbf{Estimar componente sazonal}

A componente sazonal de $S_t$ é estimada por meio de algum dos métodos
mencionados. Para a maioria dos métodos precisamos informar o período sazonal
$m$.

\subsubsection{Passo 5}

\textbf{Calcular a componente residual}

Por fim calcula-se a componente residual $R_t$ como

$$ R_t = y_t - T_t - S_t $$

A série está enfim descomposta

$$ y_t = R_t + T_t + S_t $$

\subsection{Qualificação de tendência e sazonalidade}

A presença de tendência ou sazonalidade é tipicamente visível imediatamente.
Dúvida sobre a presença ou não dessas propriedades pode surgir se o sinal
estiver imerso em ruído. Se houver presença de raízes unitárias curtos
intervalos de tempo podem ser ambíguos quanto ao determinismo de sua tendência,
isso é, uma tendência estocástica pode ser confundida com tendência
determinística. Nesses cenários é interessante detectar a presença desses
componentes por meio de testes estatísticos ou análise de correlalograma.

\subsection{Quantificação de tendência e sazonalidade}

\subsubsection{Força}

A decomposição de uma série temporal em componentes isolados de tendência,
sazonalidade e resíduos permite que a intensidade de tendência e sazonalidade
sejam quantificáveis de forma elegante. Essa intensidade de tendência ou
sazonalidade é tipicamente chamada de força.

Para séries com forte tendência é esperado que a componente de tendência
contenha considerável variância. Uma forma interessante de quantificar a força
da tendência de uma série, proposta por Hyndman et. al.~\cite{athana}, parte
da observação da variância adicional introduzida pela adição do termo de
tendência ao residual:

$$ \frac{Var(R_t)}{Var(T_t + R_t)} $$

Espera-se que essa razão seja pequena para séries com alta tendência, isso é,
haja introdução de considerável variância pela adição da componente de
tendência no denominador. Podemos então definir a força $F_T$ da tendência de
uma série como

$$ F_T = max\left(0, 1 - \frac{Var(R_t)}{Var(T_t + R_t)}\right) $$

Observe que $F_T \in real$ limitado entre 0 e 1.

De forma identicamente análoga podemos definir a força da sazonalidade de uma
série como

$$ F_S = max\left(0, 1 - \frac{Var(R_t)}{Var(S_t + R_t)}\right) $$

Essas medidas são úteis ao oferecer uma interface quantitativa aos atributos
tipicamente qualitativos de tendência e sazonalidade. É importante deixar
claro que essas medidas necessitam que a série seja decomposta em suas
componentes de tendência e sazonalidade, possivelmente dificultando sua
aplicação.

\section{Modelos Estacionários}

\subsection{O Modelo Estacionário como um Filtro Linear}

Segundo Box e Jenkins~\cite[capítulo~1.2.1]{box} os modelos estacionários que
serão abordados nas seguinte seções foram idealizados por (YULE, 1927) como
filtros lineares que modelam a série temporal de interesse por meio do
processamento de ruído branco. A modelagem apropriada do sinal em questão se
resume então ao apropriado ajuste da função de transferência $\psi$ desse
filtro a partir dos dados observados, isso é, o apropriado posicionamento de
zeros e/ou polos.

A função de transferência $\psi$ é definida no domínio do tempo por meio do
operador de atraso $L$ ao invés de $z^{-1}$. Como no contexto de filtros
digitais podemos pensar em um plano $L$, em analogia com o plano $Z$, e chegar às
exatas mesmas conclusões sobre o efeito do posicionamento de polos e zeros na
estabilidade (inclusive marginal), invertibilidade e resposta em frequência do
filtro. A diferença relevante é que o operador de atraso no domínio $Z$,
$z^{-1}$, é definido de forma inversa à referência do plano. Essa inversão não
acontece no plano $L$, levando a uma inversão das propriedades conhecidas.

É possível também descrever certos tipos de modelos-filtros como \emph{Finite
Impulse Response}(FIR) ou \emph{Infinite Impulse Response}(IIR), dependendo
a equação de recorrência, levando às propriedades conhecidas desses tipos de
filtros.

\begin{figure}
    \centering
    \begin{tikzpicture}[node distance=4.5cm,
        every node/.style={fill=white, font=\sffamily}, align=center]

        \node (system)      [activityStarts]              {Sistema LTI\\$\psi(t)$};
        \node (input)       [process, left of=system]     {$\varepsilon_t$};
        \node (output)      [process, right of=system]    {$y_t$};

        \draw[->]     (input) -- (system);
        \draw[->]     (system) -- (output);

    \end{tikzpicture}
    \vspace{.6cm}
    \caption{Representação de série temporal como modelo linear generalizado}
    \label{fig:white_noise_LTI}
\end{figure}


A figura~\ref{fig:white_noise_LTI} torna clara a interpretação traçada. Para
tornar o paralelo mais claro podemos inicialmente pensar em um sinal de entrada
$\varepsilon[n]$ ( ruído branco) transformado por uma resposta ao impulso
$h[n]$, causal de tamanho $m$, no sinal $y[n]$ pela seguinte soma de convolução

$$ y[n] = \sum^{m}_{k=0} h[n-k]\varepsilon[k] =  \sum^{m}_{k=0} \varepsilon[n-k]h[k]$$

Introduzindo o operador de atraso $L$ no somatório

$$ y[n] = \sum^{m}_{k=0} L^{k}h[k] \varepsilon[n]$$

observamos que temos um polinômio em $L$ cujos coeficientes correspondem aos
da resposta ao impulso do filtro. Podemos então, chaveando para notação de
séries temporais, reescrever a convolução como

$$ \sum^{m}_{k=0} \psi(k)L^k\varepsilon_{t} = \sum^{m}_{k=0} \psi(k)\varepsilon_{t-k}$$

Expandindo as somas temos

$$ y_t = \varepsilon_t + \psi_1 \varepsilon_{t-1} + \psi_2 \varepsilon_{t-2} + ... +  \psi_m \varepsilon_{t-m}$$
$$ y_t = \varepsilon_t + \psi_1 L\varepsilon_{t} + \psi_2 L^2 \varepsilon_{t} + ... +  \psi_m L^m \varepsilon_{t}$$
$$ y_t = \varepsilon_t(1 + \psi_1 L + \psi_2 L^2 + ... + \psi L^m)$$

\begin{equation}\label{eq:fir}
    y_t = \psi(L)\varepsilon_t
\end{equation}

onde $\psi(L)$ é um polinômio mônico em $L$.

Tikz Here


A interpretação de modelos estacionários como filtros tem suas limitações no
que diz respeito às possibilidades de herança de conhecimento da área de
processamento de sinais. Inicialmente podemos constatar que filtros lineares no
sentido tradicional se propõem em alterar as propriedades de um sinal
arbitrário que por si já contém informação. Isso leva à importância imediata da
função de impulso unitário como base natural de sinais em tempo discreto e a
uma enorme preocupação com a fase da resposta ao impulso do filtro devido à
facilidade de distorção de fase indesejada do sinal processado. Para o
modelo-filtro a função de impulso unitário não é tão importante porque esse
filtro é sempre excitado pelo mesmo tipo de sinal que não possui nenhum tipo de
estrutura a se preservar: ruído branco. A estrutura de fase desse sinal é menos
relevante ainda por ser aleatória. Essas diferenças tornam muita da teoria de
filtros inaplicável no contexto de séries temporais. Não há sentido de projetar
um modelo filtro de fase linear generalizada, por exemplo, se não há o que
preservar na fase do sinal de entrada e nenhuma preocupação com atraso de grupo.

Não obstante a interpretação de modelos estacionários como filtros lineares
facilita a assimilação de propriedades desses modelos por meio da linguagem de
processamento de sinais e nos leva a interessantes explorações de parte da
teoria de filtros aplicada à modelagem de séries temporais.

Por fim podemos notar que nesse contexto o filtro tradicional representa uma
função linear do processo estocástico de ruído branco ${\epsilon(t)}$ para a
saída ${Y(t)}$, qualificando a saída em si como um processo estocástico.
Conforme discutido na seção~\ref{sec:process} abordaremos as propriedades de
realizações específicas do processo ${Y(t)}$.



\subsection{Modelo Linear Generalizado}
\label{sec:glm}

É interessante expandir a discussão anterior para a definição de um modelo
linear generalizado (GLM).

Tomando a equação~\ref{eq:fir} com $m \rightarrow \infty$ temos um modelo linear
generalizado que corresponde ao processamento de ruído branco por um filtro
linear com resposta ao impulso de duração infinita:

\begin{equation}\label{eq:glm}
    y_t = \psi(L)\varepsilon_t = \sum^{\infty}_{0} \psi_m \varepsilon_t
\end{equation}

onde $\psi(L) = 1 + L + L^2 + L^3 ...$

A literatura estatística frequentemente se refere ao vetor de ruído branco
$\varepsilon_t$ como ``choques'' ou ``inovações''. O teorema de Wold
~\cite{wold} estabelece que qualquer série estacionária tem uma representação
dada pela equação~\ref{eq:glm} tal que $\sum^{\infty}_{0} \psi_k^2 < \infty$. Esse
resultado é equivalente à constatação que qualquer sinal estacionário pode ser
representado pelo processamento de ruído branco por um filtro com resposta ao
impulso quadrado somável. Outro ponto de vista do teorema do Wold pode ser
obtido analisando a variância do GLM.

$$\sigma^2_{GLM} = E\left[\left(\sum^{\infty}_{k=0} \psi_k \varepsilon_{t-k}\right)^2\right] = E\left[\sum^{\infty}_{k=0} \psi_k^2 \varepsilon_{t-k}^2\right]$$
$$\sigma^2_{GLM} = \sum^{\infty}_{k=0} \psi_k^2 E[\varepsilon_{t-k}^2]$$
$$\sigma^2_{GLM} = \sigma^2_{\varepsilon_t} \sum^{\infty}_{k=0} \psi_k^2 $$

Que implica a necessidade de finitude de $\sigma^2_{GLM}$. Construindo um
vetor $\mathbf{\psi}$ formado pelos coeficientes temos a variância representada
pelo quadrado da norma desse vetor $\norm{\psi}^2 = \sigma^2_{GLM}$.

\subsubsection{Autocorrelação}

A fim de definir a função de $\rho_{GLM}(\tau)$ de autocorrelação de um modelo
linear generalizado definimos inicialmente sua covariância

$$\gamma_{GLM}(\tau) = E\left[\left(\sum^{\infty}_{k=0} \psi_k \varepsilon_{t-k}\right)\left(\sum^{\infty}_{i=0} \psi_i \varepsilon_{t-i-\tau}\right)  \right] $$
$$\gamma_{GLM}(\tau) = E\left[\sum^{\infty}_{k=0} \sum^{\infty}_{i=0} \psi_k \varepsilon_{t-k} \psi_i \varepsilon_{t-i-\tau}\right] $$

onde temos que a esperança entre quaisquer $\varepsilon_t-a$ e $\varepsilon_t-b$
nula exceto para $a = b$, já que por definição $\varepsilon_t$ é composto de
variáveis aleatórias independentes. Estamos então interessados nos casos em que
$t-k = t-i-\tau$ tal que $i = k - \tau$. Substituindo as variáveis temos

$$\gamma_{GLM}(\tau) = E\left[\sum^{\infty}_{k=0} \psi_{k} \varepsilon_{t-k} \psi_{k-\tau} \varepsilon_{t-k}\right] $$
$$\gamma_{GLM}(\tau) = E\left[\sum^{\infty}_{k=0} \psi_{k}\psi_{k-\tau} \varepsilon_{t-k}^2\right] = \sigma^2_{\varepsilon_t}\sum^{\infty}_{k=0}\psi_{k}\psi_{k-\tau} $$

A autocorrelação é então dada por

$$ \rho_{GLM}(\tau) = \sum^{\infty}_{k=0}\psi_{k}\psi_{k-\tau} $$

\subsection{Modelo Média Móvel}

Um processo ${\mathbf{Y}(t, q)}$ é considerado de média móvel de ordem $q$ se uma
realização $y_t$ for definida pela equação~\ref{eq:ma(q)}, na qual
$\varepsilon_t$ representa a realização de um processo puramente aleatório. O
processo é tipicamente chamado de $MA(q)$.

\begin{equation}\label{eq:ma(q)}
    y_t = \varepsilon_t + \sum_{i=1}^{i=q} \beta_i \varepsilon_{t-i}
\end{equation}

Pela definição acima observamos que o modelo média móvel corresponde a um
filtro FIR excitado por ruído branco.

Um processo de média móvel, como sugerido pelo nome, é análogo a uma média
móvel de de observações anteriores de uma série temporal de ruído branco. Não
é de fato uma média móvel porque os coeficientes não necessariamente se somam a
um.

Podemos escrever a equação~\ref{eq:ma(q)} usando o operador de atraso por meio
da equação~\ref{eq:ma(q)_L}. Nessa equação observamos claramente como o modelo
$MA(q)$ é um caso particular do $GLM$.

\begin{equation}\label{eq:ma(q)_L}
    y_t = \varepsilon_t(1 + \sum_{i=1}^{i=q} \beta_{t-i} L^{i}) = \varepsilon_t \phi(L)
\end{equation}

A figura~\ref{fig:ma_time} ilustra o comportamento temporal de modelos $MA$ para
diferentes ordens.

\begin{figure}[H]
    \centering
    \includegraphics[scale=0.6]{figures/ma_time.png}
    \caption{Visualização no tempo de processos média móvel de ordens
    diferentes.}
    \label{fig:ma_time}
\end{figure}

Observamos que não temos clara identidade visual dos processos de média móvel
representados devido à natureza estocástica do sinal.

\subsubsection{Autocorrelação}

Definimos inicialmente a função de autocorrelação para o processo média móvel

$$\gamma_{MA(q)}(\tau) = E\left[\left(\varepsilon_t + \sum^{m}_{k=1} \beta_k \varepsilon_{t-k}\right)\left( \varepsilon_{t-\tau} + \sum^{m}_{i=1} \beta_i \varepsilon_{t-i-\tau}\right) \right] $$

observando que para $\tau > q$ temos $\gamma_{MA(q)}(\tau) = 0$, concluímos de
forma análoga à dedução da autocorrelação do processo linear generalizado a
seguinte forma

TODO: make case

$$\gamma_{MA(q)}(\tau) = \sigma_{\varepsilon_t}^2 \sum^{m}_{k=1} \beta_k\beta_{k-\tau} $$

A variância é dada por $\gamma(0)$ tal que

$$\sigma^{2}_{MA(q)} = \sigma_{\varepsilon_t}^2 \sum^{m}_{k=1} \beta{k}^2 $$

Se definirmos o vetor $\mathbf{\theta}$ a partir dos coeficientes do polinômio
$\theta(L)$ temos $\sigma^{2}_{MA(q)} = \sigma_{\varepsilon_t}^2\norm{\mathbf{\theta}}^2$.

Temos autocorrelação definida então como

$$\rho_{MA(q)}(\tau) = \frac{\sum^{m}_{k=1} \beta_k\beta_{k-\tau}}{\norm{\mathbf{\theta}}}$$

A função de autocorrelação de um processo média móvel exibe a interessante
propriedade de ``cortar'' após o atraso $q$, isso é, demonstrar autocorrelação
igual a zero após um atraso de número correspondente à ordem do processo. A
autocorrelação amostral de uma série temporal gerada por um processo de média
móvel tende a apresentar a mesma propriedade, apesar de ser perfeitamente
possível da autocorrelação amostral de um processo $MA(q)$ cair para zero
\emph{antes} do lag $q$~\cite{chatfield}.

A imagem~\ref{fig:ma_corr} demonstra a visualização dos correlalogramas
correspondentes aos modelos ilustrados no tempo pela figura~\ref{fig:ma_time}.
Observe que as autocorrelações são distintas de zero apenas para atrasos
iguais ou inferiores à ordem $q$ do processo.

\begin{figure}[H]
    \centering
    \includegraphics[scale=0.6]{figures/ma_corr.png}
    \caption{Visualização do correlalograma de processos média móvel de ordens
    diferentes.}
    \label{fig:ma_corr}
\end{figure}

Essa propriedade da função de autocorrelação possui consequências diretas para
modelagem de séries temporais: é possível que uma série temporal estacionária
com autocorrelações iguais a zero após atraso $k$ seja satisfatoriamente
aproximada por um modelo $MA(k)$. A função de autocorrelação amostral é usada
então como ferramenta de identificação da aplicabilidade e ordem de modelos
$MA(q)$.

\subsection{Modelo Autoregressivo}
\label{ssec:AR(p)}

Um processo ${\mathbf{Y}(t, p)}$ é considerado autoregressivo de ordem $p$ se
uma realização $y_t$ ele for definido pela equação~\ref{eq:ar(p)}, onde
$\varepsilon_t$ representa um processo puramente aleatório. O processo é
frequentemente chamado de $AR(p)$.

\begin{equation}\label{eq:ar(p)}
   y_t = \sum^{i=p}_{i=1} y_{t-i} \alpha_i + \varepsilon_t
\end{equation}

Um processo autoregressivo de ordem $p$ é caracterizado por uma dependência
entre uma amostra de instante de tempo $t$ e as amostras de instantes de tempo
$t-1$, $t-2$, ..., $t-p$. Como sugerido pelo nome a equação de diferenças
estabelece uma relação de regressão entre uma série temporal e suas versões
atrasadas no tempo.

Podemos reescrever a equação~\ref{eq:ar(p)} por meio do operador de atraso
resultando na equação.~\ref{eq:ar(p)_L}. O polinômio $\alpha(L)$ é chamado de
polinômio autoregressivo ou equação característica.

\begin{equation}\label{eq:ar(p)_L}
    y_t = \frac{\varepsilon_t}{1 - \sum_{i=1}^{i=p} L^i \alpha_i}
\end{equation}

Como exemplo temos que um processo autoregressivo de segunda ordem, isso é,
$AR(1)$, é definido pela expressão a seguir.

\begin{equation}\label{eq:ar_1}
    y_t = \frac{\varepsilon_t}{(1 - L\alpha_1)}
\end{equation}

O polinômio de operadores de atraso $1 - L\alpha_1$ recebe o nome de equação
característica. Ao expandir essa equação por divisão polinomial encontramos o
conhecido desenvolvimento de um filtro IIR.

\begin{equation}\label{eq:iir}
    y_t = \sum^{\infty}_{k=0} \alpha^k \varepsilon_{t-k}
\end{equation}

Observamos primeiramente que o parâmetro $\alpha$ deve ser tal que a soma do
lado direito da equação~\ref{eq:iir} seja convergente. Sabemos que essa
condição corresponde à estabilidade do filtro e veremos que corresponde também
à estacionariedade do sinal produzido pela filtragem. Concluímos que um sinal
produzido por um processo autoregressivo estacionário pode ser modelado pelo
processamento de ruído branco por um filtro IIR estável. Essa forma da
equação autoregressiva é claramente um caso específico do GLM.

A divisão polinomial entre a equação~\ref{eq:ar_1} e~\ref{eq:iir} é generalizada
como uma inversão do polinômio $\alpha(L)$, de forma que

$$ y_t\alpha(L) = \varepsilon_t = y_t = \alpha^{-1}(L)\varepsilon_t$$

$\alpha^{-1}(L)$ é bem definido sobre certas condições que serão discutidas na
seção~\ref{ssec:stability_invertibility}.

Uma visualização do comportamento temporal de processos autoregressivos é dada
pela figura~\ref{fig:ar_time_visualization}. Observa-se que as séries temporais
não exibem comportamento visivelmente distinguível no domínio do tempo, como
no caso das séries $MA(q)$.

\begin{figure}[H]
    \centering
    \includegraphics[scale=0.65]{figures/ar_time.png}
    \caption{Visualização de processos autoregressivos de diferentes ordens
    no domínio do tempo.}
    \label{fig:ar_time_visualization}
\end{figure}

\subsubsection{Autocorrelação}

Assumindo estacionariedade e multiplicando ambos os lados da
equação~\ref{eq:ar(p)} por $y_{t-\tau}$ podemos então calcular a esperança da
expressão para obter a autocovariância do processo para $\tau \neq 0$

$$ \gamma_{AR(p)}(\tau) = E\left[\left(\sum^{i=p}_{k=1} y_{t-k}y_{t-\tau}\alpha_k\right) + (y_{t-\tau}\varepsilon_t)\right] $$
$$ \gamma_{AR(p)}(\tau) = E\left[\left(\sum^{i=p}_{k=1} y_{t-k}y_{t-\tau}\alpha_k\right)\right] $$
$$ \gamma_{AR(p)}(\tau) = \sum^{i=p}_{k=1} \alpha_k \gamma_{AR(p)}(\tau - k) $$

Equação que igualmente satisfaz a autocorrelação $\rho(\tau)$ pela divisão de
cada um dos termos acima por $\gamma_{AR(P)}(0) = \sigma_{AR(p)}^2$.

\begin{equation}\label{eq:ar_autocorr}
    \rho_{AR(p)}(\tau) = \sum^{i=p}_{k=1} \alpha_k \rho_{AR(p)}(\tau - k), \tau > 0
\end{equation}

Obtemos então uma autoregressão de ordem $p$ nas autocovariâncas também. Para
calcular a variância do processo multiplicamos os termos da equação~\ref{eq:ar(p)}
por $y_{t}$

$$ \sigma_{AR(p)}^2 = \sigma_{\varepsilon_t}^2 + \sum^{i=p}_{k=1} \gamma_{AR(p)}(k)$$
$$ \sigma_{AR(p)}^2 = \sigma_{\varepsilon_t}^2 + \sigma_{AR(p)}^2\sum^{i=p}_{k=1} \rho_{AR(p)}(k)$$
$$ \sigma_{AR(p)}^2 - \left(\sigma_{AR(p)}^2\sum^{i=p}_{k=1} \rho_{AR(p)}(k)\right) = \sigma_{\varepsilon_t}^2 $$
$$ \sigma_{AR(p)}^2 = \frac{\sigma_{\varepsilon_t}^2}{1 - \sum^{i=p}_{k=1} \rho_{AR(p)}(k)}$$

A variância é também uma função autoregressiva da autocorrelação.

Podemos reescrever a equação~\ref{eq:ar_autocorr} em função do polinômio de
atraso (operando agora sobre $\tau$ ao invés de $t$) como

\begin{equation}\label{eq:ar_rho}
    \alpha(L)\rho_{AR(p)}(\tau) = 0
\end{equation}


Expressando o polinômio em função de suas raízes $G_i$ temos que

$$\alpha(L) = \prod_{k=1}^{p} (1 - G_i L)$$

É demonstrado em~\cite[capítulo~4]{box} que a solução geral para a equação de
~\ref{eq:ar_rho} é dada por

\begin{equation}\label{eq:rho_solution}
    \rho_{AR(p)}(\tau) = \sum_{k=1}^{p} A_k G^{\tau}_k
\end{equation}

Essa é a expressão conclusiva da autocorrelação de um processo autoregressivo
de ordem $p$.

Denotemos as raízes reais de $\alpha(L)$ por $R_h$ e as complexas conjugadas
por $C_j, C_k$. Observamos que sob a forma~\ref{eq:rho_solution} as raízes
reais $R_h$, assumindo que $|R_h| < 1$, contribuem para a autocorrelação como
exponenciais amortecidas $A_h^{\tau} R_h$. Raízes complexas conjugadas $C_j,
C_k$ contribuem com UM SENOIDE AMORTECIDO. DEDUZIR COMO!
TODO

$$ A_j^{\tau}C_j + A_k^{\tau}C_k $$

Concluímos então que a autocorrelação de um processo $AR(p)$ é uma combinação
de $N$ exponenciais e senóides amortecidos tal que $N \leq p$.

Para o caso específico de um processo estacionário $AR(1)$ temos pela
equação~\ref{eq:rho_solution} a seguinte relação de autocorrelação

$$ \rho_{AR(1)}(\tau) = \alpha \rho_{AR(1)}(\tau - 1)$$

$$ \rho_{AR(1)}(\tau) = \alpha^{\tau}$$

TODO: add AR figures here !!

\subsubsection{Equações de Yule Walker}

É possível estimar os coeficientes autoregressivos $\alpha$ desenvolvendo as
equações~\ref{eq:ar_rho}. Com $\rho(-k) = \rho(k)$ e $\rho(0) = 1$ temos

$$\rho(1) = \alpha_{1} + \alpha_{2}\rho(1) + \alpha_{3}\rho(2) ... + \alpha{p}\rho(p-1)$$
$$\rho(2) = \alpha_{1}\rho(1) + \alpha_{1}\rho(1) + \alpha_{2}\rho(2) ... + \alpha{p}\rho(p)$$
$$\vdots$$
$$\rho(p) = \alpha_{1}\rho(p-1) + \alpha_{1}\rho(-1) + \alpha_{2}\rho(-2) ... + \alpha{p}$$

Em forma matricial

\begin{gather}\label{eq:yule_walker}
  \begin{bmatrix} \rho(1) \\ \rho(2) \\ \vdots \\ \rho(p) \end{bmatrix}
  =
  \begin{bmatrix}
      1 & \rho(1) & ... & \rho(p-1) \\
      \rho(1) & 1 & ... & \rho(p) \\
      \vdots & \vdots & \ddots & \vdots \\
      \rho(p-1) & \rho(p-2) & ... & 1 \\
  \end{bmatrix}
  \begin{bmatrix} \alpha_{1} \\ \alpha_{2} \\ \vdots \\ \alpha_{p} \end{bmatrix}
\end{gather}

As equações~\ref{eq:yule_walker} são conhecidas como equações de Yule Walker,
que permitem uma estimativa de coeficientes autoregressivos por meio de uma
estimativa de coeficientes de autocorrelação. A forma mais intuitiva de estimar
o vetor $\mathbf{\alpha}$ é por meio da inversão da matriz de autocorrelações
defasadas, mas outros algoritmos como o de Durbin-Levinson podem ser utilizados.

\subsubsection{Autocorrelação Parcial}

Como mencionado na seção~\ref{ssec:partial_acorr} autocorrelação parcial
informa a correlação restante entre $y_t$ e $y_{t-\tau}$ após levar em
conta a contribuição resultante dos termos intermediários $y_{t-1}, y_{t-2}
... y_{t-\tau+1}$.

Autocorrelação parcial é formalmente definida para um processo autoregressivo
partindo da equação~\ref{eq:ar_rho}, onde tomando $p=1$ para um processo
autoregressivo de ordem 1 temos que

$$ \rho(\tau) = \alpha_1 \rho(\tau - 1) $$

Tomando $\tau = p = 1$ e adicionando um subscrito adicional em $\alpha$ para
indicar a ordem do processo autoregressivo temos

$$ \alpha_{1_{1}} = \rho(1) $$

Para $p=2$ temos

$$ \rho(\tau) = \alpha_1 \rho(\tau - 1) + \alpha_2 \rho(\tau - 2) $$

Novamente tomando $\tau = p = 2$ obtemos

$$ \rho(2) = \alpha_{1_{2}} \rho(1) + \alpha_{2_{2}} \rho(0) $$

Estamos como no caso de $p=1$ interessados no valor de $\alpha_{2_{2}}$, isso é,
$\alpha_{\tau_{\tau}}$ com $\tau=2$. Uma solução para esse valor em função
das autocorrelações $\rho$, obtida por meio das equações de Yule Walker, é

$$ \alpha_{2_{2}} = \frac{\rho(2) - \rho^2(1)}{1 - \rho^2(1)} $$

As autocorrelações parciais em função de $\tau$ são então dadas por sucessivos
$\alpha_{\tau_{\tau}}$ para $\tau = 1, 2, 3 ...$, isso é, o último coeficiente
autoregressivo $\alpha_p$ de processos autoregressivos de ordem $p$ crescentes.

A solução desses valores em função das autocorrelações de cada processo de
ordem crescente, dada por~\cite{morettin}, é generalizada como

$$ \alpha_{\tau_{\tau}} = \frac{\norm{\mathbf{P}^*_{\tau}}}{\norm{\mathbf{P}_{\tau}}} $$

Onde a matriz $\mathbf{\mathbf{P}}$ é a matriz de autocorrelações de ordem $p =
\tau$ como definida na equação~\ref{eq:yule_walker} e a matriz
$\mathbf{P}^*$ é obtida pela substituição da última coluna de
$\mathbf{P}$ pelo vetor de autocorrelações $\mathbf{\rho}$.

Essa definição de autocorrelação parcial é bem definida para processos
autoregressivos e para esse tipo de processo seu valor claramente se torna
igual a zero a partir do atraso $\tau = p$.

A autocorrelação parcial amostral é calculada da mesma forma partindo da
autocorrelação amostral como definida pela equação~\ref{eq:autocorr}. A
autocorrelação parcial amostral por sua vez é definida para séries temporais
arbitrárias. Nesse caso sucessivos modelos autoregressivos de ordem $p = \tau =
1, 2, 3..$ são ajustados à série em questão e o último coeficiente
de cada regressão é armazenado como a autocorrelação parcial amostral para o
atraso $\tau$.

Como estamos interessados em trabalhar com séries temporais a definição acima
de autocorrelação parcial amostral é o suficiente para generalizar o conceito
definido sobre processos autoregressivos para realizações únicas (séries
temporais) de processos estocásticos arbitrários.

\subsection{ARMA}
\label{ssec:ARMA}

Processos ARMA, como sugerido pelo nome, são gerados pela sobreposição de
processos AR e MA. Um processo ${\mathbf{Y}}(t)$ é considerado $ARMA(p, q)$ se
for dado pela equação~\ref{eq:arma}

\begin{equation}\label{eq:arma}
    y_t = \varepsilon_t + \sum_{i=1}^{i=q} \beta_i \varepsilon_{t-i} + \sum_{i=1}^{i=p} y_{t-i}\alpha_i
\end{equation}

Um processo $ARMA(2, 1)$ é portanto dado pela seguinte equação de recorrência.

$$ y_t = \alpha_1 y_{t-1} + \alpha_2 y_{t-2} + \varepsilon_{t} + \beta_1 \varepsilon_{t-1} $$

Em analogia aos casos anteriores o processo pode ser escrito por meio do
operador de atraso.

$$ y_t (1 - \alpha_1 L - \alpha_2 L^2) = \varepsilon_{t} (\beta_0 + \beta_1 L) $$

Podemos introduzir $\phi(L) = 1 - \alpha_1 L - \alpha_2 L$ e $\theta(L) =
\beta_0 + \beta_1 L$ tal que:

\begin{equation}\label{eq:arma_l}
    y_t = \varepsilon_t \frac{\phi(L)}{\theta(L)}
\end{equation}

A equação~\ref{eq:arma_l} é prontamente generalizada para corresponder à
equação~\ref{eq:arma} estabelecendo:

$$\phi(L) = 1 - \alpha_1 L - \alpha_2 L^2 \hdots - \alpha_p L^p$$

$$\theta(L) = 1 - \beta_1 L^1 \hdots - \beta_q L^q$$


Observamos que a forma da equação~\ref{eq:arma_l} é idêntica à de uma função de
transferência, sendo definida no domínio do tempo sobre polinômios de $L$ ao
invés de no domínio $Z$ sobre polinômios em $z$. As implicações do
posicionamento das raízes dos polinômios numerador e denominador decorrem de
maneira análoga.

Uma propriedade interessante de modelos ARMA é que processos estacionários
frequentemente podem ser modelados por modelos ARMA com menos parâmetros que
modelos AR ou MA.

\subsection{Estabilidade e Invertibilidade}
\label{ssec:stability_invertibility}

\section{Raízes Unitárias}
\label{sec:unit_roots}

Como um processo ARMA é essencialmente resultante do processamento de ruído
branco por um filtro linear com função de transferência
$\frac{\theta(L)}{\phi(L)}$ sabemos que a posição das raízes dos polinômios
$\theta(L)$ e $\phi(L)$ determina suas propriedades.

O posicionamento de raízes no círculo unitário é um caso interessante de
analisar. Se algum dos polos função de transferência de um processo ARMA
estiverem posicionados no círculo unitário diz-se que esse processo possui
raízes unitárias. Esse termo tipicamente descreve o posicionamento dos polos e
não dos zeros pelo maior efeito dessas raízes na dinâmica do sistema, mas essa
seção inclui uma breve discussão sobre o efeito de zeros unitários também.

Antes de prosseguir um processo $y_t$ $ARMA(2, 1)$ será estabelecido para
exemplificar as seguintes discussões.

Diante da recorrência

$$ y_t = \alpha_1 y_{t-1} + \alpha_2 y_{t-2} + \varepsilon_t + \beta_1 \varepsilon_{t-1}$$

temos

$$ y_t = \frac{1 + \beta_1L}{1 - \alpha_1 L - \alpha_2 L^2} \varepsilon_t $$

Uma realização desse processo para os parâmetros

$$\alpha_1 = -0.5, \alpha_2 = 0.25, \beta_1 = 0.8$$

é ilustrada pela figura~\ref{fig:ARMA2-1} e seu diagrama de polos e zeros
no plano $L$ pela figura~\ref{fig:ARMA2-1pzp}.

\begin{figure}[H]
    \centering
    \includegraphics[scale=0.5]{figures/arma_21.png}
    \caption{Visualização de realização de processo ARMA(2, 1) no tempo}
    \label{fig:ARMA2-1}
\end{figure}

\begin{figure}[H]
    \centering
    \includegraphics[scale=0.7]{figures/arma_21_pzp.png}
    \caption{Visualização de posicionamento de polos e zeros de processo
    ARMA(2, 1) em relação ao
    círculo unitário no plano L}
    \label{fig:ARMA2-1-pzp}
\end{figure}

\subsection{Raízes Unitárias no Polinômio Autoregressivo}

Inspecionemos qualitativamente a inserção presença de uma raiz unitária no
polinômio autoregressivo do processo ARMA(2, 1) apresentado. Tomando $z_t$
como o processo

$$ z_t = \frac{y_t}{(1 - L)} = \frac{1 + 0.8L}{(1 + 0.5 L - 0.25 L^2)(1 - L)} \varepsilon_t $$

Temos a seguinte relação entre $z_t$ e $y_t$

$$ (1 - L)z_t = y_t $$
$$ z_t - z_{t-1} = y_t $$
$$ \nabla z_t = y_t $$

Tal que $z_t$ represente uma integração de $y_t$.

$$ z_t = \nabla^{-1} y_t $$

Dizemos que nesse caso o processo $z_t$ é integrado de ordem um, ou $I(1)$, já
que o diferenciando uma vez temos em um processo estacionário. Observamos
imediatamente que o processo $z_t$ não é estacionário.

Podemos definir um processo integrado de ordem $d$, $I(d)$, como um processo
cuja diferenciação em $d$ vezes resulta em estacionariedade. Tal processo seria
gerado a partir de $y_t$ por meio de $d$ integrações. Essa definição implica
corretamente que um processo $I(d)$, $d > 0$, não é estacionário, já que a
condição de estacionariadade para processos ARMA é pontualmente violada.

Analisando uma realização do processo $z_t$, por meio da
figura~\ref{fig:ARMA-2-1-integrated}, observamos um típico processo de raiz
unitária, cuja não estacionariedade não é tão óbvia quanto nos casos de um
filtro ARMA instável ou presença de tendência determinística. Processos com
raíz unitária são diferença estacionários e possuem tendência determinística,
como discutido na seção~\ref{ssec:taxonomy}. Como esperado o diagrama de
polos e zeros desse modelo inclui um polo adicional correspondente à raiz
unitária.

\begin{figure}[H]
    \centering
    \includegraphics[scale=0.5]{figures/arma_21_integrated.png}
    \caption{Visualização de realização de processo ARMA(2, 1) com introdução
    de raiz unitária no tempo}
    \label{fig:ARMA2-1-integrated}
\end{figure}

\begin{figure}[H]
    \centering
    \includegraphics[scale=0.7]{figures/arma_21_integrated_pzp.png}
    \caption{Visualização de posicionamento de polos e zeros de processo
    ARMA(2, 1) com introdução de raiz unitária em relação ao
    círculo unitário no plano L}
    \label{fig:ARMA2-1-integrated-pzp}
\end{figure}

Processos de raízes unitárias são especialmente interessantes por sua
capacidade de representar fenômenos do mundo real de forma eficiente. Logo
serão reconhecimentos como processos da classe ARIMA.

No processamento de sinais determinísticos sistemas lineares com raiz unitária
levam ao caso de estabilidade marginal, onde a resposta ao impulso do sistema é
um sinal de potência ao invés de um sinal de energia. A ideia de uma resposta
ao impulso de energia infinita sob presença de polos integradores manifesta-se
no contexto de sinais aleatórios quando é dito que processos de tendência
estocástica (de estacionariedade diferenciável, como discutido na seção
~\ref{ssec:taxonomy}) são afetados irreversivelmente por eventos perturbadores.
A ``lembrança'' desses eventos é justamente a resposta de duração infinita do
evento perturbador, apropriadamente modelado por funções impulso ou degrau.
Essa interpretação é importante no campo de análise de intervenção, que procura
incorporar efeitos de eventos perturbadores reais em modelos de séries
temporais.

\subsection{Raízes Unitárias no Polinômio Média Móvel}\label{ssec:ma_roots}

A presença de uma raiz unitária no polinômio de média móvel de um filtro ARMA
gera efeitos menos dramáticos em sua dinâmica. Antes de tudo nota-se que a
introdução de raízes unitárias de médias móveis tornam um sistema não
inversível, já que sua inversão  tornaria essa raiz um polo.

As figuras~\ref{fig:ARMA2-1-diff} e~\ref{fig:ARMA2-1-diff-pzp} ilustram uma
realização do processo $y_t$ com adição de raiz unitária de média móvel e o
diagrama de polos e zeros resultante, respectivamente.

TODO: check this @ later time !!

Observamos que a série com raiz de média móvel adicional aparenta ter uma
distribuição de potência mais enviesada para altas frequências, de forma
recíproca ao de raiz unitária autoregressiva, em que há introdução de
componentes de baixa frequência. Isso será discutido apropriadamente no
capítulo ~\ref{chap:spectral_analysis}.


\begin{figure}[H]
    \centering
    \includegraphics[scale=0.5]{figures/arma_21_diff.png}
    \caption{Visualização de realização de processo ARMA(2, 1) derivado
    no tempo}
    \label{fig:ARMA2-1-diff}
\end{figure}

\begin{figure}[H]
    \centering
    \includegraphics[scale=0.7]{figures/arma_21_diff_pzp.png}
    \caption{Visualização de posicionamento de polos e zeros de processo
    ARMA(2, 1) derivado em relação ao círculo unitário no plano L}
    \label{fig:ARMA2-1-diff-pzp}
\end{figure}

Tomando o processo $w_t$ como

$$ w_t = (1 - L)y_t = \frac{1 - 0.2L + L^2}{1 + 0.5L - 0.25L^2}\varepsilon_t$$

Temos $w_t = y_t - y_{t-1} = \nabla y_t$ tal que $w_t$ seja a derivada de
$y_t$. Vemos que ato de diferenciar uma série, em uma tentativa de introduzir
estacionariedade por exemplo, corresponde à introdução de uma raiz unitária no
polinômio de média móvel. Conseguimos dessa forma compreender a indução de
estacionariedade em uma série não estacionária com raiz unitária autoregressiva
por meio de sua derivação como a sobreposição de $d$ zeros aos $d$ polos
integradores do processo original. Isso é ilustrado por uma comparação das
figuras~\ref{fig:ARMA2-1-integrated-pzp} e~\ref{fig:ARMA2-1-diff-pzp}, onde
vemos que a diferenciação da série $z_t$ introduziria um zero unitário de forma
a ``cancelar'' seu polo unitário.

A presença de raízes unitárias de média móvel podem indicar que a série foi
diferenciada mais vezes que necessário ou que diferenciação é a operação
inadequada para indução de estacionariedade, isso é, a série não é diferença
estacionária.

Um exemplo é a série tendência estacionária $m_t$ a seguir:

$$ m_t = \mu + \eta t + \varepsilon_t $$

$$ \nabla m_t = \varepsilon_t - \varepsilon_{t-1} + \eta $$
$$ \nabla m_t = (1 - L)\varepsilon_t + \eta $$

Em $\nabla m_t$ temos uma raiz unitária no polinômio autoregressivo, resultando
em uma série não inversível. A tendência determinística poderia ter sido
removida por regressão resultando em um processo estacionário e inversível.


\section{Modelo ARIMA}
\label{sec:ARIMA}

Também chamados de modelos ARMA Integrados modelos ARIMA são essencialmente
modelos ARMA com tentativas de remoção de não estacionariedade. Assumindo que a
série em questão seja diferença-estacionária  o modelo ARIMA induz
estacionariedade ajustando um modelo ARMA com raízes unitárias adicionais. A
expectativa é que o processo original contenha $d$ polos unitários que serão
então sobrepostos por $d$ zeros unitários, resultando em processo estacionário.

A distinção entre o modelo ARMA e ARIMA é a substituição de $y_t$ por $\nabla^d
y_t$.

Definindo $w_t = \nabla^d y_t$ para $d = 0, 1, 2, ...$ temos a definição de um
processo $ARIMA(p, d, q)$ dada pela equação~\ref{eq:arima}. Observe que a
equação é idêntica à equação~\ref{eq:arma} com $w_t$ ao invés de $y_t$ e que a
saída é $w_t$, demandando uma integração para previsão de $y_t$. O nome do
modelo referencia essa operação de reconstrução de $y_t$.

\begin{equation}\label{eq:arima}
    w_t = \varepsilon_t  + \sum_{i=1}^{i=q} \beta_i \varepsilon_{t-i} + \sum_{i=1}^{i=p} w_{t-i}\alpha_i
\end{equation}

Podemos escrever um modelo $ARIMA(p, d, q)$ por meio do operador de atrasos
por meio da equação~\ref{eq:arima_l}, onde:

$$\phi(L) = 1 - \alpha_1 L - \alpha_2 L^2 \hdots - \alpha_p L^p$$

$$\theta(L) = 1 - \beta_0 - \beta_1 L^1 \hdots - \beta_q L^q$$

Nessa forma o paralelo entre modelos ARMA e ARIMA se torna mais claro.

\begin{equation}\label{eq:arima_l}
    y_t = \varepsilon_t \frac{\theta(L)}{\phi(L)} \frac{1}{(1-L)^d}
\end{equation}

O modelo ARIMA segue a metologia Box-Jenkins de modelagem, assumindo
diferença-estacionariedade e tentando induzir estacionariedade por meio de
sucessivas operações de diferenças. Pela discussão acima é claro que um modelo
$ARIMA(p, 0, q)$ corresponde a um modelo $ARMA(p, q)$.

O leitar agora reconhece o sistema da figura~\ref{fig:ARMA2-1-integrated-pzp}
como ARIMA(2, 1, 1).

\section{Modelos de Sazonalidade}\label{sec:seasonality_models}

\subsection{Diferenciação Sazonal}

O operador de diferença sazonal $\nabla_s$ estende o operador de diferenças
para subtração entre elementos não adjacentes e age sob uma série temporal
$\mathbf{y}_t$ da seguinte forma:

$$ \nabla \mathbf{y}_t = (1 - L^s)\mathbf{y}_t = \mathbf{y}_t - L^s\mathbf{y}_t = \mathbf{y}_t - \mathbf{y}_{t-s} $$

O operador mantém as propriedades da diferença simples e pode ser empregado em
ordens elevadas da mesma forma:

$$\nabla_4^2 y_t = \nabla_4 \nabla_4 y_t = \nabla_4 (y_t - y_{t-4}) = y_t - 2y_{t-4} + y_{t-8} $$

\subsection{SARIMA}\label{ssec:SARIMA}

O modelo ARIMA pode ser estendido de forma natural para acomodar padrões de
sazonalidade. O modelo SARIMA (Seasonal ARIMA) parte da observação que saídas
de sinais com alta sazonalidade podem possuem alta correlação com saídas
anteriores em uma distância corresponde ao padrão de sazonalidade, como
constatado no capítulo 1.

O modelo SARIMA então estabelece, em adição à operação de diferenciação do
modelo ARIMA, uma operação de diferenciação sazonal ao definir a componente
$w_t$ da seguinte forma:

$$ w_t = \nabla^D_{s}\nabla^d  y_t $$

Como na definição de $w_t$ para o modelo ARIMA temos uma diferenciação
tradicional de ordem $d$ seguida de uma diferenciação sazonal de período $s$
e ordem $D$.

Como exemplo um $w_t$ definido em função de $d=1$, $D=1$ e $s=24$ assume a
seguinte forma.

$$ w_t =  \nabla^1_{12} \nabla^1 y_t = \nabla^1_{2} (y_t - y_{t-1}) $$
$$ w_t =  (y_t - y_{t-24}) - (y_{t-1} - y_{t-25}) $$

Em seguida componentes autoregressivos e de média móvel são introduzidos com
atrasos em múltiplos de $s$. O modelo final é descrito como $SARIMA(p, d, q)
(P, D, Q)_{s}$, com $P$ e $D$ referenciando os componentes $AR$ e $MA$
sazonais. Para ilustrar melhor essa notação podemos analisar a expressão de um
modelo específico para depois generalizar.

Um modelo $SARIMA(1, 0, 1)(2, 1, 1)_{12}$ é dado pela seguinte expressão.

$$ w_t = \varepsilon_t + \overbrace{\alpha_1 w_{t-1}}^{p = 1} + \overbrace{
\alpha_2 w_{t-12}}^{P=1} + \overbrace{\alpha_3 w_{t-24}}^{P=2} +
\overbrace{\beta_1 \varepsilon_{t-1}}^{q=1} + \overbrace{\beta_2
\varepsilon_{t-12}}^{Q=1}$$

A escrita desse modelo por meio do operador de atraso é mais clara.

$$  w_t = \varepsilon_t \frac{\beta_1 L^1 + \beta_2 L^{12}}{(1 - \alpha_1 L^1 + \alpha_2 L^{12} + \alpha_3 L^{24})}  $$

Podemos então generalizar o modelo na forma do operador de atraso por meio da
equação~\ref{eq:sarima}.

\begin{equation}\label{eq:sarima}
    w_t = \varepsilon_{t} \frac{\theta_q(L)\theta_Q(L)}{\phi_p (L) \phi_P (L^s)}
\end{equation}

Com os polinômios de atraso definidos como nos casos anteriores.

A diferenciação sazonal inicial do modelo SARIMA é um dos exemplos mais simples
de um passo essencial no processamento de séries com padrões sazonais: a
modelagem do padrão de sazonalidade em si, que de fato é um campo próprio com
livros como~\cite{x11}. Diferenciação sazonal é limitada em sua capacidade de
expressar múltiplos padrões de sazonalidade. Outra forma mais flexível de
modelagem de sazonalidade será abordada na subseção seguinte.

\subsection{Variáveis de Fourier}

Versões sazonais de modelos ARIMA, como SARIMA, são pouco eficientes para
modelar padrões sazonais com as seguintes características:

\begin{itemize}
    \item Múltiplas periodicidades~\cite{athana}
    \item Períodos múltiplos fracionários do tempo de amostragem~\cite{hyndman_weekly}.
    \item Curto tempo de amostragem (de um dia ou inferior)~\cite{athana}
    \item Períodos maiores que algumas centenas do tempo de
    amostragem~\cite{hyndman_long_season}.
\end{itemize}

Sinais elétricos quase sempre possuem as duas últimas propriedades, tornando a
exploração de métodos alternativos indispensável para este trabalho.

A inclusão de variáveis de Fourier como regressores é capaz de modelar
padrões sazonais como esses de forma mais flexível. Essas variáveis exógenas
aos ao modelo são somas de senos e cossenos, ambos reais, que oscilam em
múltiplos de uma frequência fundamental definida como a frequência do padrão
sazonal que se deseja modelar.

Para modelar um padrão de período $m$ tomamos as variáveis dadas pela série
$F_t$

$$ F_t = \sum^K_{k=1} \left(\alpha_k \frac{sen(2\pi k t)}{m} + \beta_k\frac{cos(2\pi k t)}{m}\right) $$

onde harmônicos do período sazonal $m$ são incluídos pelo incremento de $k$.
Para incluir múltiplos períodos sazonais as variáveis são generalizadas para
diferentes valores de $m$ como a seguir:

$$ F_t = \sum^M_{i=1}\sum^K_{k=1} \left(\alpha_k \frac{sen(2\pi k t)}{m_i} + \beta_k\frac{cos(2\pi k t)}{m_i}\right) $$

Temos alguns parâmetros não regressores: $K$, a ordem dos harmônicos, $m_i$, os
períodos dos padrões que se deseja modelar. $M$ é simplesmente o número de
padrões. A determinação de $m_i$ pode ser realizada por conhecimento a priori
do fenômeno em questão ou métodos quantitativos como análise espectral dos
dados amostrais. Já $K$ é melhor determinado por métodos de seleção de modelo
como AIC e BIC, discutidos na seção~\ref{sec:validacao}.

A série exógena pode então ser incluída como informação auxiliar em um modelo
ARIMA resultando em um modelo sazonal $y_t$ em função de $(p, d, q, m_i, k)$:

$$ y_t = \varepsilon_t \frac{\theta(L)}{\phi(L)} \frac{1}{(1-L)^d} + F_t $$

$$ y_t = \varepsilon_t \frac{\theta(L)}{\phi(L)} \frac{1}{(1-L)^d} + \sum^M_{i=1}\sum^K_{k=1} \left(\alpha_k \frac{sen(2\pi k t)}{m_i} + \beta_k\frac{cos(2\pi k t)}{m_i}\right) $$

	\chapter{Teoria Espectral Univariada}\label{chap:spectral_analysis}

\section*{Introduction}

O seguinte capítulo discute a representação espectral de séries temporais
univariadas. São exploradas transformações lineares e não lineares.

O desenvolvimento de uma representação espectral para uma realização
teoricamente infinita de um processo estocástico se inicia no reconhecimento da
impossibilidade de uma transformada de Fourier desse tipo de sinal, seguido de
uma apresentação do teorema de Wiener Khinchin. O resultado da representação
desenvolvida, conhecido como densidade de potência espectral, é interpretado.
Em seguida, restringindo a classe de sinais estacionários para realizações de
processos ARMA conseguimos derivar expressões fechadas para a densidade
de potência espectral, resultado importante em teoria de
estimação\cite{estimation_theory}.

Em seguida inicia-se uma discussão sobre representações não lineares por meio
de uma generalização natural da função de autocorrelação e subsequentemente a
classe de distribuições de Cohen. O problema geral dessa classe de
representações e a principal proposta para sua resolução nos leva às
transformadas conhecidas como \emph{Smoothed Pseudo Wigner Ville
Distributions} (SPWVD).

\section{Stationary Analysis}

Wide sense stationary signals are, by definition (section
~\ref{sec:stationarity}), power signals. Since the Fourier transform of a
signal is well defined only if is has finite energy stationary signals do not
have a Fourier representation in the traditional sense, the exception being
quasi-periodic signals which can be represented by a Fourier series. In order
to develop a spectral representation of stationary time series we must define
the concept of a power spectral density and conclude that this function is
proportional to the square magnitude of a hypothetical Fourier transform.

\subsubsection{Power Spectral Density}

We begin by stating Parseval's theorem, in which $F\{\}$ represents the Fourier
transform.

$$ E = \int^{\infty}_{-\infty} |x(t)|^2 dt = \frac{1}{2\pi} \int^{\infty}_{-\infty} |F\{x(t)\}(\omega)|^2 d\omega $$

Extending this definition to signal power gives us

$$ P = \lim_{T \to \infty} \frac{1}{2T}\frac{1}{2\pi} \int^{T}_{-T}|F\{x(t)\}(\omega)|^2 d\omega$$

Note that even though $F\{x(t)\}$ is not well defined here the above relations
still hold if $|F\{x(t)\}|^2$ is defined in a different manner, which will be
done shortly.

The signal power can be rewritten denoting the Fourier transform of $x(t)$ by
$X(\omega)$ as

$$ P = \lim_{T \to \infty} \frac{1}{2T}\frac{1}{2\pi} \int^{T}_{-T}|X(\omega)|^2 d\omega $$

Where $\lim_{T \to \infty}\frac{1}{2\pi} \frac{1}{2T} |X(\omega)|² $
is recognized as a density function. The power spectral denUma interpretaçãosity function is
finally defined as

$$ S_{x}(\omega) = \lim_{T \to \infty}\frac{1}{2\pi} \frac{1}{2T} |X(\omega)|² $$

This function's name is pretty explanatory of its interpretation:
$S_{x}(\omega)$ represents the contribution of $x(t)$s frequency components in
$\omega + d\omega$ to the overall signal power. As mentioned, for this function
to make any sense we must define $|X(\omega)|^2$, which will be done
presently

\subsection{Wiener-Khinchin Theorem}

The Wiener-Khinchin theorem can be developed as follows.

$$ |X(\omega)^2| = X(\omega)X^*(\omega) = F(F^{-1}(X(\omega))*(F^{-1}(X^*(\omega))) = F(x(t) * x^*(-t)) = F(x(t) * x(-t))$$

Examining the right-most part of this equality we observe that the function
which is being Fourier-transformed corresponds to the convolution of $x(t)$
with a mirrored version of itself. This is precisely the definition of
autocorrelation. Assuming ergodicity we can now express the squared magnitude
of the Fourier transform of $x(t)$ as the Fourier transform of is autocorrelation
function.

$$|X(\omega)_T|^2 = \frac{1}{2\pi}\int_{-T}^{T} \rho(t)e^{-j \omega t}dt$$

This results is known as the Wiener-Khinchin theorem and it allows for a
well-defined power density spectrum for stochastic signals.

Note that since the autocorrelation of a signal is even its Fourier transform
is real-valued, which is consistent with our notion of a squared magnitude.

\subsection{Spectrum of ARMA processes}

By taking the square magnitude Z transform of the general ARMA recurrence
relationship (~\ref{ssec:arma_l}) we obtain the transfer function

$$ H(z) = \frac{1 + \sum_{i}^{q} b_k z^{-k}}{1 + \sum_{i}^{q} a_k z^{-k}} $$

which is excited by white noise to generate a realization of and ARMA
process. We can now express the power spectral density of an ARMA process as
follows

% $$ S_{ARMA}(\omega) = |H(z)|^2 S_{\varepsilon} $$
% \begin{equation}\label{eq:arma_spectrum}
%      S_{ARMA}(\omega) = \frac{\sigma^2 |1 + \sum_{k=1}^{q} b_k e^{-j\omega k}|^2}{2\pi|1 + \sum_{k=1}^{p} a_k e^{-j\omega k}|^2} $$
% \end{equation}

This definition can be used as a means of parametric spectral estimation: the
parameters are estimated in the time domain and used by the relationship
above to estimate the spectrum.

We will now visualize the spectra of some ARMA processes.

\subsubsection{MA(1)}

\begin{figure}[H]
    \centering
    \includegraphics[scale=0.7]{figures/ma_1_spectrum.png}
    \caption{Spectrum of an MA(1) process with
    $\protect \beta_1 = -0.5$}
    \label{fig:ma_1_spectrum}
\end{figure}

\subsubsection{AR(1)}

\begin{figure}[H]
    \centering
    \includegraphics[scale=0.7]{figures/ar_1_spectrum_1.png}
    \caption{Spectrum of an AR(1) process with
    $\protect \alpha_1 = 0.8$}
    \label{fig:ar_1_spectrum_1}
\end{figure}

\begin{figure}[H]
    \centering
    \includegraphics[scale=0.7]{figures/ar_1_spectrum_2.png}
    \caption{Spectrum of an AR(1) process with
    $\protect \alpha_1 = -0.8$}
    \label{fig:ar_1_spectrum_2}
\end{figure}

\subsubsection{AR(2)}

\begin{figure}[H]
    \centering
    \includegraphics[scale=0.7]{figures/ar_2_spectrum.png}
    \caption{Spectrum of an AR(1) process with
    $\protect \alpha_1 = 0.5$ and $\protect \alpha_2 = -0.25$}
    \label{fig:ar_2_spectrum}
\end{figure}

\subsubsection{ARMA(4, 3)}

\begin{figure}[H]
    \centering
    \includegraphics[scale=0.7]{figures/arma_4_3_spectrum.png}
    \caption{Spectrum of an ARMA(4, 3) process}
    \label{fig:ar_4_3_spectrum}
\end{figure}


\subsection{Effect of Unit Roots on ARMA Spectra}

As mentioned in subsection~\ref{ssec:ma_roots} the introduction of unit roots
to the moving average or auto regressive polynomials of an ARMA process has
clear effects on its spectral content. The spectral effects of these operations
can be understood by a pole zero plot, sample time domain or closed form
spectral analysis.

By analyzing the pole zero plot of an ARMA system it is evident how integration
affects the system's transfer function: since a pole is introduced at
$e^{j\omega}=0$ energy is supplied to low-frequency signal components.
Conversely we can understand differencing, which is the introduction of a
zero at $e^{j\omega}=0$, as a suppression of low-frequency components,
resulting in a signal with higher overall frequency. This is precisely the
idea behind integrating filters.

In the time domain the spectral effect of differencing can be seen by comparing
figures~\ref{fig:ARMA2-1} and~\ref{fig:ARMA2-1-diff}. The differenced series
clearly has more energy distributed towards high frequency components. This is
in fact true for all signals: the act of time domain differencing biases the
spectral content towards higher frequencies. The reverse is also true: by
integrating a signal we see that it becomes smoother.

TODO: complete this !!!

Finally the most rigorous understanding is achieved by inspection of equation
~\ref{eq:arma_spectrum}. From this equation it is evident that an additional
root on the numerator increases the high frequency content and an additional
root on the denominator increases the low frequency content.

We now present the differenced version of the MA(1) spectrum from figure
~\ref{fig:ma_1_spectrum}. Note that the signal is essentially high pass
filtered, as mentioned.

\subsubsection{Differenced MA(1)}

\begin{figure}[H]
    \centering
    \includegraphics[scale=0.7]{figures/diff_ma_1_spectrum.png}
    \caption{Spectrum of an differenced MA(1) process with
    $\protect \beta_1 = -0.5$}
    \label{fig:diff_ma_1_spectrum}
\end{figure}


\section{Linear Non-Stationary Analysis}

From the previous section we can conclude that if a time series is stationary
its spectral representation via power spectral density is uniquely determined
by the Fourier transform of its autocorrelation function. The natural extension
for a spectral representation of non-stationary processes is a time-varying
power spectral density since its autocorrelation function is also time-varying.

The idea of a time-varying spectral representation gives rise to the so called
time-frequency analysis methods. We initially explore methods that maintain
linearity.

\subsection{Short Time Fourier Transform}

The short time Fourier transform is the most intuitive approach to a proposal
of time frequency representation.

\subsection{Wavelet Transform}

\subsubsection{Continuous}

\subsubsection{Discrete}

\section{Non-Linear Representations}

There is, indeed, a way to maximize the time-frequency resolution trade-off
inherent to time-frequency representations (TFR). This is done by the
introduction of non linearity through the Fourier transform of the
instantaneous autocorrelation function. As will be seen presence of non
linearity results in cross terms that limit the quality of the representation.
Attempts to dampen these cross terms lead to the more general Cohen's class of
distributions.

\subsection{Instantaneous autocorrelation function}

The instantaneous autocorrelation function is actually just the autocorrelation
function of a non stationary signal written in a specific format.
Interestingly the term autocorrelation function has become strongly understood
as a function of sample lag $\tau$, which is the case for stationary signals,
instead of a function of $t_1$ and $t_2$. We initially rewrite the general
autocorrelation function $R_{xx}$ of a signal $x(t)$

$$ R_{xx}(t_1, t_2) = E[x(t_1)x(t_2)] $$

We can also write $R_xx$ as a function of a single moment $t$ and a lag $\tau$

$$ R_{x}(t, \tau) = E[x(t)x(t-\tau)] $$

Which is slightly more natural for computations. Note that if $x(t)$ is
stationary the dependence on time is removed because $R_{xx}$ has the same
value for all $t$.

A small adjustment in notation leads to

$$ \mathcal{R}_{x}(t, \tau) = E\left[x\left(t - \frac{\tau}{2}\right)x\left(t + \frac{\tau}{2}\right)\right] $$

With a new symbol to indicate that we have finally arrived at the instantaneous
autocorrelation function $\mathcal{R}_{xx}$.

A non stationary TFR is now natural. Since the
Wiener-Khinchin theorem states that the spectrum of a stationary signal is the
Fourier transform of its autocorrelation function we can in an analogous manner
assume that a spectral representation of a non stationary process will be given
by the Fourier transform along the $\tau$ axis of the instantaneous
autocorrelation function. This leads to the Wigner-Ville distribution.

\subsection{Wigner-Ville Distribution}

We define the Wigner-Ville distribution (WVD) as

$$ \mathcal{W}_{x}(t, f) =  \int_{-\infty}^{\infty} \mathcal{R}_{x}(t, \tau) e^{-j\omega \tau}d\tau$$

$$ \mathcal{W}_{x}(t, f) =  \int_{-\infty}^{\infty} x\left(t - \frac{\tau}{2}\right)x\left(t + \frac{\tau}{2}\right) e^{-j\omega \tau}d\tau$$

This natural representation can be thought of as an instantaneous power density
spectrum. It is known that the WVD optimizes the time-frequency resolution
trade-off\cite{tfr_comparison} which is exactly our goal in the development of
more elaborated TFRs. We will now see that this is not without its problems.

\subsubsection{Cross-terms}

By construction the Wigner-Ville distribution is a quadratic representation. By
the quadratic superposition principle~\cite{quadratic_freq_representation} we
know that if $x(t) = \mu x_1(t) + \lambda x_2(t)$ the WVD representation of
$x(t)$ will be given by

$$ \mathcal{W}_{x} = \mu^2\mathcal{W}_{x_1} + \lambda^2\mathcal{W}_{x_2} + 2(\lambda\mu)^2(\mathcal{W}_{x_1 , x_2})$$

Where $\mathcal{W}_{z, y}$ represents the cross-WVD from $z$ to $y$. Since any
real signal of relevant complexity is a linear combination of the $cos(t)$
$sin(t)$ basis we can expect a considerable introduction of unwanted
information from the cross-WVD components, referred to as cross-terms. This is
the infamous cross-terms problem attributed to the WVD and is one of the
reasons that despite its precision in time and frequency resolution it is not
an ideal choice for the TFR for most signals.

The cross-terms are known to exhibit high frequency patterns~\cite{martin_lol},
leading to the idea that the WVD could be filtered in order to be more
representative of its auto-terms. The different ways in which it is possible
and useful to filter the WVD generates what is known as the Cohen's class of
distributions.

\subsection{Smoothed Pseudo Wigner Ville Distributions}

Most members of Cohen's class of distributions are essentially filtered
versions of the WVD~\cite{}. A particularly useful case is known as the
Smoothed Pseudo Wigner Ville Distribution (SPWVD)

$$ SPWVD(t, f) = \int_{-\infty}^{\infty}\int_{-\infty}^{\infty} h_t (t - \tau) h_f(f - \phi)d\tau d\phi$$

in which $h_t$ denotes the filter applied along time and $h_f$ along frequency.

If the filters are well designed, which is a data-driven process in some
cases~\cite{}, cross-term suppression is sufficiently successful such that
the use of the SPWVD over simpler non-stationary TFR is justified.

	% \input{text/2-textuais/5-testes_estatisticos}
	\chapter{Aplicação}
\label{chap:aplicacoes}

\section{Definição de Problema}

- Não pode falar muito devido ao sigilo
- Deixar claro que a intenção é a extração de características

\subsection{Problemas de operacionalização}

- Tem que funcionar em tempo real, qualquer modelo envolvido tem que evoluir
no tempo. Mencionar filtragrem bayesiana como solução. Mencionar que modelos
complexos como os SOTA da literatura dificilmente vão funcionar. linearidade
seria melhor
- Processamento do sinal deve ser facilmente reprodutível. Mencionar DVC como
solução.

\section{Descrição dos Dados}

Mencionar alta frequência, multiplicidade de séries e sla

\section{Análise de Propriedades das Séries Temporais}

Aplicar testes de sazonalidade, tendência, estacionariedade.

Levantar hipóteses sobre possível resolução:
    - decomposição
    - representação tempo frequêcia por meio de DWT?

\section{Decomposição Linear}

\subsection{Modelagem de Tendência}

\subsection{Modelagem de Sazonalidade}

\subsubsection{Padrão Sazonal Médio}

\subsubsection{Variáveis de Fourier}

\subsubsection{Análise de Resíduos}

\subsubsection{Modelagem de Resíduos}

\section{Representação em Tempo Frequência}

\subsection{Espectrograma}

\subsection{Distribuição de Wigner Ville}

\subsection{Transformada Contínua de Ondaletas}

\subsection{Transformada Discreta de Ondaletas}

\section{\emph{Pipeline} de Processamento de Modelagem}

mostrar pipeline dvc, talvez até montar um tikz legal e tal

\section{Resultados}

\section{Análise de Estabilidade Residual por meio Diagrama de Polos e Zeros}

	\input{text/2-textuais/7-conclusao}

	%Elementos pós-textuais
	\bibliography{text/3-pos-textuais/referencias}

\end{document}
