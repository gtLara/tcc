Séries temporais como uma categoria de dados ganhou imensa importância com o
advento de \emph{big data} por representar essencialmente qualquer tipo de
sistema que evolui no tempo. Grande parte da teoria envolvida no campo
responsável pelo estudo desse tipo de dado é idêntica ou muito semelhante a
conceitos de sistemas lineares e processamento de sinais. O objetivo deste
trabalho é explorar o poder analítico de uma interpretação de teoria clássica
de séries temporais por meio de conceitos de sistemas lineares e processamento
de sinais, no que diz respeito à resolução de problemas da área. No capítulo 2
é estabelecida uma série de conceitos fundamentais, como álgebra de operadores
e estacionariedade, de forma intuitiva. No capítulo 3 são discutidos elementos
e procedimentos tradicionais da teoria de séries temporais univariadas
utilizando terminologia e conceitos de sistemas lineares, levando à uma
delimitação das possibilidades de comunicação entre as duas áreas. No capítulo
4, são abordadas representações espectrais espectrais de séries estacionárias e
não estacionárias novamente por meio de teoria de sinais e sistemas, levando a
uma interpretação elegante do efeito em frequência das operações de integração
e derivação. Por fim, no capítulo 5, um problema prático de detecção de
anomalias em séries temporais multivariadas a partir de poucos dados é tratado
utilizando do \emph{framework} conceitual desenvolvido em capítulos anteriores.
É demonstrado que essa abordagem é eficiente mesmo na condição de escassez de
dados, que tipicamente inviabiliza algoritmos mais populares de aprendizado de
máquina. Por fim, é travada uma breve discussão sobre a integração da
metodologia considerada neste trabalho com estes algoritmos.

\palavraschave{Análise de Séries Temporais. Sistemas Lineares.
Processamento de Sinais de Tempo Discreto. Representações em tempo-frequência.}
\newpage
