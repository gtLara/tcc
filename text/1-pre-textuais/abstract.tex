Time series, as a data category, has become more relevant with the rise of big
data since they represent essentially any type of system that evolves over
time. Much of the theory involved in the analysis of time series is very
similar or indistinguishable from signal processing and linear system concepts.
This work's objective is to explore the analytical power of an interpretation
of classical time series theory through the lens of linear systems, as it
pertains to the solution of practical problems. Chapter 2 establishes a series
of fundamental concepts, such as operator algebra and stationarity, in an
intuitive manner. Chapter 3 discusses traditional elements and procedures
of univariate time series analysis using terminology and concepts of linear
systems, leading to a delimitation of the limits of communication between the
two areas. In chapter 4 stationary and non stationary spectral representations
of time series are presented, once more in the language of linear systems,
leading to an elegant interpretation of the frequency domain effects of
integration and differencing. In chapter 5 a practical multivariate anomaly
detection problem is tackled using the theoretical framework developed
beforehand. Is is demonstrated that this approach is efficient even in the
absence of large volumes of data, condition which typically prohibits the use
of modern machine learning algorithms. A final discussion on the integration of
of the present methodology with these algorithms is developed.

% Separe as Keywords por ponto
\keywords{Time series analysis. Linear Systems. Discrete-time signal processing.
Time Frequency Representations}
\newpage
