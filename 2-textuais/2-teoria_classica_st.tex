\part{Teoria Clássica de Séries Temporais}
\label{cap:teorica_classica_series_temporais}


\chapter*{Introdução}
\label{sec:teorica_classica_series_temporais_introducao}

\chapter{Definições e Propriedades}
\label{sec:teorica_classica_series_temporais_definicoes}

\section{Série Temporal}\label{ssec:definition}

Uma série temporal é um conjunto de observações realizadas sequencialmente no
tempo, indexadas de acordo com o momento em que foram observadas. As observações
são tipicamente de um mesmo processo, tal que cada uma represente o resultado
de uma amostragem. A natureza do processo subjacente é relevante para análise e
modelagem de qualquer série temporal, como será visto nas próximas seções.

Assume-me, na linguagem de~\cite{hamilton}, que um conjunto de amostras
$\mathbf{y}_t = (y_1, y_2, y_3 \dots y_T)$ pode ser interpretado como um
segmento finito de uma sequência duplamente infinita:

$${\mathbf{y}}_{t=-\infty}^{\infty} = ({\dots, y_{-1},y_0, \overbrace{y_1, y_2, y_3, \dots, y_T}^{\text{Série Observada}}, y_{T+1}, y_{T+2}}\dots)$$

\vspace{1cm}

Apesar de parecer pouco tangível, de fato qualquer série observada é
satisfatoriamente descrita dessa forma. Em um contexto industrial, por exemplo,
o valor de uma variável de processo é zero até a planta ser construída e entrar
em operação, assume valores representativos ou não da dinâmica de interesse
(assumindo valores irrelevantes quando a planta não está em operação plena), e
tendendo ao infinito retorna a zero quando a planta for desativada.

Essa interpretação de uma série temporal é importante ao implicitamente
insinuar que o processo existe em um intervalo temporal mais abrangente do que
o observado. É portanto necessário se questionar sobre quanto os
dados representam o processo analisado e em quais intervalos de tempo.

TODO: Séries temporais são inerentemente diferentes de dados de corte
transversal por representarem amostras discretas de um mesmo processo ao invés
de amostras aleatórias de uma população.

A forma mais natural de analisar uma série temporal é visualizar seus valores
no tempo, como ilustra a imagem~\ref{fig:example}.

\begin{figure}[H]
    \centering
    \includegraphics[scale=0.6]{figuras/white_noise.png}
    \caption{Visualização no tempo de ruído branco}
    \label{fig:example}
\end{figure}


\section{Operador de Atraso(\emph{Lag})}

É importante introduzir o operador de atraso que será referido nesse texto
como operador \emph{lag}.

Dadas as séries $\mathbf{y}_t = (y_1, y_2, y_3 \dots y_T)$ e
$\mathbf{x}_t = (x_0, x_1, x_2 \dots x_{T-1})$ tal que

$$ \mathbf{y}_t = \mathbf{x}_{t-1}$$

isso é,

$$ y_1 = x_0 $$
$$ y_2 = x_1 $$
$$ \vdots $$
$$ y_T = x_{T-1} $$

Podemos definir $\mathbf{x}_t$ em função de $\mathbf{y}_t$ como:

$$ \mathbf{x}_t = L\mathbf{y}_t $$

tal que

$$\boxed{\mathbf{y}_{t-1} = L\mathbf{y}_t}$$

Observamos que o operador de atraso atrasa uma série temporal em uma unidade
de tempo. Uma breve divagação matemática~\cite{hamilton} permite definir o
operador com propriedades muito semelhantes à multiplicação, como
associatividade, comutatividade e distribuição. Para atrasar múltiplas unidades
de tempo temos que:

\vspace{1cm}

$$L(L(\mathbf{y_t})) = L(\mathbf{y}_{t-1}) = \mathbf{y}_{t-2} = L^2 \mathbf{y}_{t}$$

\vspace{1cm}

de forma que

$$ L^n {\mathbf{y}} =  \mathbf{y}_{t-n}$$

\vspace{1cm}

Um uso importante do operador, decorrente de suas propriedades algébricas, é
exemplificado na seguinte expansão

$$ (aL^2 + bL^3) \mathbf{y}_{t} =  a\mathbf{y}_{t-2} + b\mathbf{y}_{t-3}$$

Conhecimento do operador de atraso é importante para compreender a literatura
de séries temporais e facilita comunicação objetiva de análises cotidianas.
O presente trabalho usa do operador para descrever uma série de modelos.

Como nota final é importante mencionar que alguns livros~\cite{chatfield}
~\cite{stoffer} usam a letra $B$ para denotar o operador de atraso e que na
maior parte dos recursos \emph{online} o operador é referido por seu nome em
inglês, \emph{lag}.

A analogia entre o operador de atraso e a variável complexa $e^{-j\omega} = z^{-1}$
é clara, com a relevante diferença que $z^{-1}$ atrasa um sinal em uma unidade
de tempo se a operação for realizada no domínio $z$ enquanto o operador de
atraso atua diretamente no domínio do tempo. Essa característica permite que
filtros com equações de recorrência complexas sejam representados de forma
compacta no domínio do tempo por meio de polinômios de atraso.

Uma propriedade interessante decorrente da equivalência entre $z^{-1}$ e $L$ é
que pode se pensar em um plano $L$ cuja análise é igualmente informativa à do
plano z, notando que o espaço é de certa forma invertido. Uma análise da
posição dos polos de um sistema representado por meio de um polinômio em $L$
conclui que o sistema é instável se tais polos estiverem fora do círculo unitário,
contrário do que conhecemos do plano z.

Por fim vale mencionar que alguns autores\cite{aguirre}\cite{algemlol}
usam a notação $q^{-1}$ para esse operador.

\section{Operador de Diferença}\label{sec:diff}

O operador de diferenças $\nabla$ ou  $\Delta$ é o equivalente discreto da
operação contínua de diferenciação e opera sob uma série temporal
$\mathbf{y}_t$ da seguinte forma:

$$ \nabla \mathbf{y}_t = (1 - L)\mathbf{y}_t = \mathbf{y}_t - L\mathbf{y}_t = \mathbf{y}_t - \mathbf{y}_{t-1} $$

O operador possui propriedades de associatividade e distribuição, tal que

$$ \nabla^2 \mathbf{y}_t = \nabla(\nabla(\mathbf{y}_t)) = \nabla(\mathbf{y}_t - \mathbf{y}_{t-1}) = \nabla \mathbf{y}_t - \nabla \mathbf{y}_{t-1} = \mathbf{y}_t - 2 \mathbf{y}_{t-1} + \mathbf{y}_{t-2} $$

\section{Tendência(\emph{Trend})}

A variação do valor esperado do processo gerador de uma série temporal ou a
variação da média móvel de suas observações é denominado tendência. A imagem
~\ref{fig:trend} ilustra uma série com tendência linear. Observa-se que ao
longo do tempo a média das observações cresce linearmente. Tendências de séries
reais frequentemente seguem um perfil logarítmico~\cite{chatfield}, como na
figura~\ref{fig:log_trend}. Nesse caso uma transformação exponencial da série,
isso é, a aplicação de uma função exponencial a cada observação, tornaria a
tendência linear.

\begin{figure}[H]
    \centering
    \includegraphics[scale=0.6]{figuras/trend.png}
    \caption{Série temporal com tendência linear}
    \label{fig:trend}
\end{figure}

\begin{figure}[H]
    \centering
    \includegraphics[scale=0.6]{figuras/log_trend.png}
    \caption{Série temporal com tendência logarítmica}
    \label{fig:log_trend}
\end{figure}

\section{Sazonalidade}\label{sec:seasonality}

A variação periódica de média móvel das observações de uma série temporal é
denominada sazonalidade. Em séries no contexto de finanças sazonalidade
frequentemente segue ciclos de calendário como anual, mensal, semestral, etc.
No contexto mais amplo de séries temporais sazonalidade apresenta período
arbitrário, apesar da linguagem em torno dessa propriedade estar muito
associada aos períodos anteriormente mencionados.

Um exemplo de série temporal com sazonalidade é ilustrado na
figura~\ref{fig:seasonality}.

\begin{figure}[H]
    \centering
    \includegraphics[scale=0.6]{figuras/seasonality.png}
    \caption{Série temporal com sazonalidade senoidal de período arbitrário}
    \label{fig:seasonality}
\end{figure}

Na presença de tendência sazonalidade pode ser considerada aditiva, se sua
variação for constante em torno da tendência, e multiplicativa, se sua variação
depender o valor da tendência. Exemplos de sazonalidade aditiva e
multiplicativa são dados pelas figuras~\ref{fig:add_seasonality} e
~\ref{fig:mult_seasonality}, respectivamente. Discernir entre os dois tipos de
sazonalidade é importante para modelagem.

\begin{figure}[H]
    \centering
    \includegraphics[scale=0.6]{figuras/add_seasonality.png}
    \caption{Série temporal com sazonalidade aditiva}
    \label{fig:add_seasonality}
\end{figure}

\begin{figure}[H]
    \centering
    \includegraphics[scale=0.6]{figuras/mult_seasonality.png}
    \caption{Série temporal com sazonalidade multiplicativa}
    \label{fig:mult_seasonality}
\end{figure}

\section{Autocorrelação}

Autocorrelação surge no contexto de séries temporais como a correlação amostral
entre uma série $\mathbf{y}_t$ e sua versão atrasada em um número arbitrário de
$k$ atrasos $L^k \mathbf{y}_t$. Podemos definir a autocorrelação $r_k$ de um
atraso $k$ diretamente a partir da definição de correlação amostral por meio da
equação~\ref{eq:autocorr} sob as seguintes premissas:

\begin{enumerate}
    \item A média $\bar{y}$ de $\mathbf{y}_t$ é constante no tempo
    \item O número de observações  $N$ de $\mathbf{y}_t$ é suficientemente
    grande ($N \approx 100$)
\end{enumerate}

\vspace{1cm}

\begin{equation}\label{eq:autocorr}
    r_k = \frac{\sum_{t=1}^{N-k}(y_t - \bar{y})(y_{t+k}-\bar{y})}{\sum_{t=1}^{N}(y_t - \bar{y})^2}  , \hspace{1cm} k = 0, 1, 2, \dots
\end{equation}

\vspace{1cm}

\subsection{Autocorrelação Parcial}

É interessante mencionar a existência de autocorrelação parcial nessa seção,
mas sua definição será apresentada na seção~\ref{ssec:AR(p)} seguindo a
prática de~\cite{chatfield}.

Dissertar mais!

\subsection{Correlalograma}\label{ssec:correlalogram}

Um correlalograma é um gráfico de barras representativo da autocorrelação em
$k$ amostras de uma série temporal ($k=0, 1, 2, \dots$), de forma que a
primeira barra represente a autocorrelação entre $y_t$ e si mesmo (sempre igual
à 1), a segunda entre $y_t$ e $y_{t-1}$, a terceira entre $y_t$ e $y_{t-2}$, e
assim por diante. O correlalograma da série visualizada pela
figura~\ref{fig:trend} é ilustrado na figura~\ref{fig:correlalogram}.

\begin{figure}
    \centering
    \includegraphics[scale=0.5]{figuras/corr_trend.png}
    \caption{Visualização de correlalograma de série com tendência linear.
    Observe que as autocorrelações decaem lentamente ao decorrer dos atrasos,
    comportamento típico de tendências determinísticas.}
    \label{fig:correlalogram}
\end{figure}

\begin{figure}
    \centering
    \includegraphics[scale=0.5]{figuras/corr_seasonality.png}
    \caption{Visualização de correlalograma de série com sazonalidade.
    Observe que o padrão senoidal da série é reproduzido nas autocorrelações,
    apesar do alto valor de atraso necessário para o padrão se tornar visível.}
    \label{fig:corr_season}
\end{figure}

A figura~\ref{fig:correlalogram} informa um intervalo de relevância dado por um
sombreamento vermelho. Qualquer valor de autocorrelação dentro desse intervalo
é estatisticamente insignificante e pode ser considerado igual a zero.

O correlalograma é uma ferramenta indispensável em análise de séries temporais
para tarefas como detecção de estacionariedade, identificação de sazonalidade,
análise de resíduo, engenharia de características, escolha de modelo e
identificação de ruído branco(seção~\ref{sec:white_noise}), entre outros.

É importante mencionar que o correlalograma de uma série com tendência
determinística, como a da figura~\ref{fig:correlalogram}, apresenta o
comportamento observado de autocorrelações altas com pouca atenuação ao longo
dos atrasos. De forma análoga o correlalograma de uma série com sazonalidade
apresenta periodicidade que reproduz seu padrão temporal, como ilustra a
figura~\ref{fig:corr_season}, correlalograma da série da
figura~\ref{fig:seasonality}. Ambos correlalogramas são informativos sobre suas
respectivas séries até certo ponto: eles informam simplesmente que as séries
apresentam tendência e sazonalidade. Para analisar tais séries de forma mais
produtiva, a fim de elaborar um possível modelo, por exemplo, é importante que
a série seja estacionária (seção~\ref{sec:stationarity}). É inclusive afirmado
em \cite{chatfield}\cite{hamilton}\cite{box}, que um correlalograma só faz
sentido se a série associada for estacionária, observação mais geral e rigorosa
das restrições de uso da equação~\ref{eq:autocorr}.

\section{Correlação Cruzada}

\section{Raízes Unitárias}\ref{sec:unit_roots}

\section{Estacionariedade}\label{sec:stationarity}

Uma série temporal $\mathbf{x}_t$ gerada por um processo $\mathbf{X}(t)$ é
considerada estacionária se atender às seguintes três condições:

\begin{enumerate}
    \item $E(\mathbf{X}(t)) = \mu$
    \item $Var(\mathbf{X}(t)) = \sigma^2$
    \item $Cov[\mathbf{X}(t), \mathbf{X}(t+\tau)] = \gamma(\tau)$
\end{enumerate}\vspace{.5cm}

Que podem ser interpretadas da seguinte forma

\begin{enumerate}
    \item A média da série $\mathbf{x}_t$ é constante ao longo do tempo
    \item A variância da série $\mathbf{x}_t$ é constante ao longo do tempo
    \item A série não apresenta sazonalidade
\end{enumerate}\vspace{.5cm}

Uma interpretação mais formal do ponto 3 pode ser resumida pela necessidade  da
autocorrelação de $\mathbf{x}_t$ depender apenas do atraso e não do tempo.

\begin{figure}[h]
    \centering
    \includegraphics[scale=0.6]{figuras/stationary.png}
    \caption{Conjunto de séries demonstrando diferentes níveis de
    estacionariedade.}
    \label{fig:stationarity}
\end{figure}

Um bom exemplo de graus de estacionariedade em séries temporais dado por
Athanasopoulos e Hyndman~\cite{athana} é ilustrado pela
figura~\ref{fig:stationarity}. As séries das figuras~\ref{fig:stationarity}.a,
~\ref{fig:stationarity}.e e~\ref{fig:stationarity}.i demonstram clara
tendência, sendo portanto não estacionárias. As séries das figuras
~\ref{fig:stationarity}.d, ~\ref{fig:stationarity}.h, ~\ref{fig:stationarity}.i
são igualmente não estacionárias por apresentarem clara sazonalidade, enquanto
a série da figura~\ref{fig:stationarity}.g aparenta ser sazonal mas apresenta
picos aperiódicos de intensidade muito distinta, sendo considerada estacionária
pelos autores. O caso da série da figura~\ref{fig:stationarity}.g é um
complicado por aparentar também violar o requisito de variância constante. Para
concluir assim como os autores que a série é estacionária é necessário mais do
que uma investigação visual; o texto que discute a série leva em conta seu
processo gerador~\cite{athana}.

As séries das figuras~\ref{fig:stationarity}.c e ~\ref{fig:stationarity}.f são
aparentemente não estacionárias no intervalo observado por demonstrarem
variação em sua média móvel mas podem ser um caso de raízes unitárias.

Por eliminação temos que apenas as séries ilustradas pelas
figuras~\ref{fig:stationarity}.b e ~\ref{fig:stationarity}.g são estacionárias,
o caso de ~\ref{fig:stationarity}.b contendo uma clara anomalia.

O exemplo de análise de estacionariedade da figura~\ref{fig:stationarity}
demonstra a imprecisão da abordagem visual para essa tarefa. Torna-se
necessário o estabelecimento de procedimentos mais objetivos para detecção de
estacionariedade (seção ~\ref{ssec:stationarity_tests}).

\subsection{Importância de Estacionariedade}

Estacionariedade é uma propriedade desejável de se observar em uma série
temporal para fins analíticos e de modelagem.

TODO: Reposicionar seguinte parágrafo

Como mencionado na seção~\ref{ssec:correlalogram}, análise de uma série por
meio de correlalograma é informativa apenas se a série for estacionária, uma
vez que o correlalograma de uma série não estacionária comunica essencialmente
apenas isso - informação sobre tendência (as autocorrelações permanecerem
altas através de vários \emph{lags}) ou sazonalidade (as autocorrelações
refletem em magnitude a periodicidade da série temporal). O correlalograma de
uma série não estacionária dificilmente seria útil para identificar sua não
estacionariedade caso isso não seja imediatamente visível no domínio do tempo.

Há algumas formas de compreender como estacionariedade colabora para a
modelagem bem sucedida de uma série temporal.

Podemos pensar na propriedade de estacionariedade como um tipo de estrutura de
dependência. Se as amostras $X_1, X_2, \dots, X_N$ de um conjunto $\mathbf{X}$
forem independentes entre si temos formas interessantes de modelar a função
geradora de $\mathbf{X}$ como o teorema do limite central, lei dos grandes
números, etc. Há apenas uma forma de um conjunto amostral ser independente mas
muitas formas de ser dependente, tornando difícil o estabelecimento de recursos
eficientes para modelagem geral de processes dependentes. Séries temporais
sendo naturalmente observações de processos dependentes
(seção~\ref{ssec:definition}) é interessante definir estruturas de dependência
que permitam o uso de tais recursos. Estacionariedade é uma estrutura de
dependência que permite aplicar propriedades úteis de independência em séries
temporais. Abordando a mesma ideia mais intuitivamente podemos pensar no
seguinte exemplo: se um processo possui valor esperado e variância constante e
não apresenta sazonalidade podemos por meio da lei dos grandes números estimar
seu valor esperado e variância com cada vez mais confiança a partir da média e
variância amostral, respectivamente. O mesmo argumento intuitivo se estende
analogamente para o caso de aprendizado de máquina, no qual por meio de teoria
de aprendizado estatístico é possível argumentar que uma série estacionária é
``mais fácil'' de aprender.

De forma mais quantitativa o teorema de decomposição de Wold~\cite{chatfield}
conclui que qualquer série temporal estacionária pode ser representada pela
seguinte combinação linear

$$\mathbf{y}_t = \sum_{j=0}^\infty b_j Z_{t-j} + \eta_t$$

No qual $\eta$ representa uma série determinística e $Z_t$ um processo
puramente aleatório (seção~\ref{sec:white_noise}). O leitor reconhecerá parte da expressão
acima como um processo $MA(\infty)$ (seção~\ref{ssec:MA}). Esse resultado tem
como consequência a importante conclusão que qualquer série estacionária é
possivelmente aproximável por um modelo MA e portanto, via invertibilidade,
modelos AR e ARMA (seção~\ref{sec:stationary_models}).

Por fim estacionariedade permite o uso de uma série de modelos que serão
discutidos na seção sobre modelos estacionários. Esses métodos são bem
compreendidos e implementados, facilitando sua interpretação, uso e
sustentação.

\subsection{Categorias Básicas de Não Estacionariedade}\label{ssec:taxonomy}

Como extensão do argumento sobre estruturas de dependência na seção anterior
podemos afirmar que, sendo estacionariedade um padrão de dependência, temos
infinitas formas de não estacionariedade, retornando ao caso de dependência
generalizada. É interessante identificar nesse universo de dependência padrões
de séries não estacionárias que são facilmente transformadas em séries
estacionárias. (Incluir essa divagação?)

Uma série temporal com presença de tendência determinística, como ilustrada na
figura~\ref{fig:trend}, pode ser representada pela seguinte expressão:

$$  y_t = e_t + f(t) + \varepsilon_t  \hspace{1cm}\text{onde} \hspace{.4cm}\varepsilon_t \sim \hspace{.2cm}\text{i.i.d.} \hspace{.2cm}\mathcal{N}(0, \sigma^2)$$

Na qual $e_t$ representa uma série estacionária, $f(t)$ uma função
determinística do tempo e $\varepsilon_t$ ruído
branco(seção~\ref{sec:white_noise}). Nota-se que $f(t)$ é uma função
monotônica arbitrária tal que $y_t$ seja uma série não estacionária. No caso da
figura~\ref{fig:trend} temos $f(t)$ linear e na figura ~\ref{fig:log_trend}
logarítmica. Uma série temporal demonstrando esse tipo de não estacionariedade
é considerada \textbf{tendência-estacionária}, uma vez que simplesmente
removendo a tendência $f(t)$ temos estacionariedade. Isso pode ser feito de
várias formas, talvez com maior simplicidade diferenciando a série. Métodos
mais sofisticados incluem decomposição ETS (seção~\ref{sec:decomposition}) e
regressão com finalidade de modelar $f(t)$ de forma que o resíduo represente
uma a série estacionária $e_t + \varepsilon_t$.

\begin{figure}[h]
    \centering
    \includegraphics[scale=0.5]{figuras/random_walk.png}
    \caption{Visualização de caminhada aleatória com $y_0=5$}
    \label{fig:random_walk}
\end{figure}

Uma série com presença de tendência estocástica pode ser classificada de
maneira semelhante. O exemplo mais simples de tal série é gerada por um
passeio aleatório, definido pelo seguinte processo, visualizado pela figura
~\ref{fig:random_walk}:

$$  y_t = y_{t-1} + \varepsilon_t  \hspace{1cm}\text{onde} \hspace{.4cm}\varepsilon_t \sim \hspace{.2cm}\text{i.i.d.} \hspace{.2cm}\mathcal{N}(0, \sigma^2)$$

Por meio de um desenvolvimento recursivo do processo podemos escrever:

$$ y_t = (y_{t-2} + \varepsilon_{t-1}) + \varepsilon_{t} $$
$$ y_t = ((y_{t-3} + \varepsilon_{t-2}) + \varepsilon_{t-1}) + \varepsilon_{t} $$
$$ \vdots $$
$$ y_t = \sum_{j=0}^{N-1} \varepsilon_{t-j} + y_0$$
\vspace{1cm}

Resultado a partir do qual a não estacionariedade de $y_t$ se torna evidente,
uma vez que

$$ var(y_t) = \sigma^2 t $$

Além da covariância ser dependente do tempo.

Uma forma simples de tornar a série estacionária é diferenciá-la em primeira
ordem, isso é, aplicar o operador de diferença primeira(seção~\ref{sec:diff}):

$$ \nabla y_t = y_t - y_{t-1} $$
$$ y_t - y_{t-1} = \varepsilon_t$$
$$ \nabla y_t =  \varepsilon_t$$

Sabemos pela seção~\ref{sec:white_noise} que ruído branco é um processo
estacionário.

A caminhada aleatória é denominada uma série \textbf{diferença-estacionária}
pelo fato da operação de diferença introduzir estacionariedade. Essa é uma
forma tão comum de não estacionariedade que a ideia de ``diferenciar uma
série antes de fazer qualquer coisa'' é proeminente entre profissionais de
dados, apesar de que geralmente necessita-se apenas de estacionariedade
\footnote{essa prática é parcialmente justificada considerando que a maior
parte das séries temporais ``reais'' são não estacionárias e frequentemente
diferencialmente estacionárias}.

É igualmente possível que uma série diferença-estacionária seja estacionária em
sua $n$-ésima diferença, tal que estacionariedade seja observada por uma
operação de diferenças de ordem $n$. A ideia de tirar sucessivas diferenças
até atingir estacionariedade é fundamental no método de Box-Jenkins, por
exemplo.

Séries diferença-estacionárias apresentam raízes unitárias e os dois termos são
frequentemente usados nos mesmos contextos.

Podemos resumir as definições das categorias de não estacionariedade abordadas
nessa seção assim como suas implicações como segue:

\begin{enumerate}
    \item \textbf{Estacionariedade em Tendência}: Uma série é considerada
        tendência-estacionária se apresentar uma tendência determinística. No
        caso de anomalias ou eventos de perturbação séries com esse tipo de
        tendência retornam ao valor da tendência ao longo do tempo,
        ``esquecendo'' o evento perturbador. Esse tipo de série se torna
        estacionária pela remoção da tendência determinística, processo
        realizado por meio de regressão da tendência, por diferenciação, por
        decomposição, etc.
    \item \textbf{Estacionariedade Diferenciável}: Uma série é considerada
        diferença-estacionária ou se apresentar uma tendência estocástica. No
        caso de anomalias ou eventos de perturbação séries com esse tipo de
        tendência são irreversivelmente afetadas,
        ``lembrando'' do evento perturbador. Esse tipo de série se torna
        estacionária por diferenciação em ordem $n$. Possui raízes unitárias
        e é frequentemente discutida nessa linguagem.

\end{enumerate}

TODO: retomar estacionarização de séries em capítulo ou apêndice dedicado?

\subsection{Detecção de Estacionariedade}\label{ssec:stationarity_tests}

Espera-se que as discussões das seções anteriores tenham estabelecido motivação
suficiente para o início de uma exploração de métodos de detecção de
estacionariedade mais objetivos que simples análise visual. As seguintes
subseções abordam esses métodos, a maioria dos quais são testes paramétricos.

\subsubsection{Análise de Correlalograma}

Apesar de pouco formal e altamente sujeito a erro alguns autores e muitos
praticantes~\cite{chatfield}~\cite{metcalfe} apontam para a possibilidade de
identificar não estacionariedade a partir de análise visual do correlalograma
de uma série temporal. Como mencionado na seção~\ref{ssec:correlalogram}
séries que violam os requisitos de estacionariedade por tendência
determinística ou sazonalidade apresentam correlalogramas distintos -
autocorrelações altas e persistentes no caso de tendência determinística e
autocorrelações periódicas ao longo de valores altos de atraso no caso de
sazonalidade. No entanto séries cuja não estacionariedade for facilmente
detectada via correlalograma também são frequentemente notavelmente não
estacionárias via simples análise temporal.

\subsubsection{Teste de \emph{Dickey-Fuller}}

Os testes de \emph{Dickey-Fuller} testam a hipótese nula de presença de raízes
unitárias no processo gerador de uma série temporal com a hipótese alternativa
de estacionariedade.

O teste mais simples de \emph{Dickey-Fuller} assume que o processo gerador da
série temporal em questão é dado pela equação~\ref{eq:ad}, na qual
$\varepsilon$ é ruído branco.

\begin{equation}\label{eq:ad}
    y_t = \phi y_{t-1} + \varepsilon_t
\end{equation}

A hipótese nula do teste é a presença de raízes unitárias em $\mathbf{y}_t$,
isso é, $\phi = 1$, e a hipótese alternativa é $\phi < 1$, correspondente à
estacionariedade.

$$
\begin{cases}
    H_0: \phi = 1, \text{não estacionariedade (raízes unitárias, possível diferença-estacionariedade)} \\
    H_1: \phi < 1, \text{estacionariedade}
\end{cases}
$$

Em seguida $Ly_t$ é subtraído de ambos os lados da equação~\ref{eq:ad},
resultando no desenvolvimento a seguir:

$$ y_t - y_{t-1} = \phi y_{t-1} - y_{t-1} + \varepsilon $$
$$ \nabla y_t = (\phi - 1) y_{t-1}  + \varepsilon $$
$$ \nabla y_t = \delta y_t  + \varepsilon $$

As hipóteses do teste são agora reformuladas para as seguintes

$$
\begin{cases}
    H_0: \delta = 0, \text{não estacionariedade (raízes unitárias, diferença-estacionariedade)} \\
    H_1: \delta < 0, \text{estacionariedade}
\end{cases}
$$

A estatística de teste é computada e comparada com um valor crítico proveniente
da distribuição de \emph{Dickey-Fuller} (geralmente sob $p=0.5$) para rejeição
ou não da hipótese nula.

O modelo da série temporal, dado explicitamente pela equação~\ref{eq:ad}, pode
ser alterado para testar raízes unitárias com constante e com constante e
tendência determinística no tempo por meio das equações~\ref{eq:ad_constant} e
~\ref{eq:ad_constant_trend}, respectivamente. Note que ambas as equações são
apresentadas em função de $\delta$. É mais comum testar por apenas raízes
unitárias, uma vez que uma análise subjetiva visual, por correlalograma ou
decomposição juntamente de remoção de tendência ou sazonalidade é tipicamente
realizada antes de um teste estatístico de estacionariedade.

\begin{equation}\label{eq:ad_constant}
    \nabla y_t = \delta y_{t-1} + u_t + a_0
\end{equation}

\begin{equation}\label{eq:ad_constant_trend}
    \nabla y_t = \delta y_{t-1} + u_t + a_0 + a_1 t
\end{equation}

O teste aumentado de \emph{Dickey-Fuller}(ADF) modela o processo de forma mais
geral, incluindo na equação~\ref{eq:ad} termos representativos de processos
estacionários arbitrários. O teste aumentado é projetado para remover
autocorrelação do processo de validação de hipótese. De forma identicamente
análoga ao teste de DF temos expansões do ADF para incluir constantes e
tendências como nas equações~\ref{eq:ad_constant} e
~\ref{eq:ad_constant_trend} apesar de que, como no teste de DF, essas
variações são pouco usadas. A estatística de teste do ADF é negativa, isso é,
quanto menor seu valor maior a rejeição da hipótese nula de não
estacionariedade (maior certeza de estacionariedade).

Como em qualquer teste de hipótese um valor $p$ maior que $0.05$ indica falha
em rejeitar a hipótese nula, nesse caso correspondendo à impossibilidade de
constatar estacionariedade. Um valor $p$ menor ou igual a $0.05$ indica
rejeição da hipótese nula, correspondendo à conclusão que a série sob análise
é estacionária.

Na prática o teste mais usado é o ADF que é uma simples extensão mais robusta
do teste de \emph{Dickey-Fuller}. Implementações eficiente e populares existem
para \verb+R+ e \verb+Python+.

\subsubsection{Teste \emph{Kwiatkowski-Phillips-Schmidt-Shin} (KPSS)}

O teste KPSS desempenha uma função semelhante ao ADF com a relevante
diferença de inerentemente modelar uma tendência linear no tempo por meio da
equação~\ref{eq:KPSS}. Seu desenvolvimento matemático é análogo porém mais
trabalhoso que o caso do ADF e será portanto omitido.

\begin{equation}\label{eq:KPSS}
    y_t = \phi y_{t-1} + \varepsilon_t + \beta t
\end{equation}

Com $\varepsilon_t$ representando ruído branco. O teste em seguida define as
seguintes hipóteses:

$$
\begin{cases}
    H_0: \text{a série apresenta tendência-estacionariedade} \\
    H_1: \text{a série apresenta raízes unitárias}
\end{cases}
$$

Como em qualquer teste de hipótese um valor $p$ maior que $0.05$ indica falha
em rejeitar a hipótese nula, nesse caso correspondendo à impossibilidade de
constatar que a série não apresenta tendência estacionariedade, boa evidência
de que a série é tendência-estacionária. Um valor $p$ menor ou igual a $0.05$
indica rejeição da hipótese nula, correspondendo à conclusão que a série sob
análise possui raízes unitárias e é portanto não estacionária.

TODO: reescrever esse parágrafo !!

Observe que há uma diferença crítica: a alternativa nula não postula não
estacionariedade, como no caso do ADF, mas sim tendência-estacionariedade
(seção ~\ref{ssec:taxonomy}) decorrente diretamente da inclusão de tendência
linear no modelo da equação~\ref{eq:KPSS}. A diferença principal da alteração
da hipótese nula é que o KPSS é usado para investigar presença de
estacionariedade sob uma tendência determinística (tendência-estacionariedade)
e o ADF (tipicamente) de estacionariedade propriamente dita. O KPSS é bem
implementado em \verb+R+ e \verb+Python+.

\section{Ruído Branco}\label{sec:white_noise}

Uma série temporal $\mathbf{x}_t$ gerada por um processo $\mathbf{X}(t)$ é
considerada ruído branco ou um processo puramente aleatório se atender às
seguintes três condições:

\begin{enumerate}
    \item $E(\mathbf{X}(t)) = 0$
    \item $Var(\mathbf{X}(t)) = \sigma^2$
    \item $Cov[\mathbf{X}(t), \mathbf{X}(t+\tau)] = 0$
\end{enumerate}\vspace{.5cm}

Que podem ser interpretadas da seguinte forma

\begin{enumerate}
    \item A média da série $\mathbf{x}_t$ é nula ao longo do tempo
    \item A variância da série $\mathbf{x}_t$ é constante ao longo do tempo
    \item Não há correlação entre as amostras de $\mathbf{x}_t$
\end{enumerate}\vspace{.5cm}

Observa-se que ruído branco é um caso específico de estacionariedade, se
diferenciando pela especificação do valor esperado e autocorrelação entre
quaisquer amostras em zero. A compreensão da definição e capacidade de
identificação de ruído branco é importante para análise de resíduos, detalhado
na seção~\ref{ssec:residual_analysis}.

\subsection{Detecção de ruído branco}
